\documentclass{sig-alternate-10pt}

\special{papersize=8.5in,11in}
\setlength{\pdfpageheight}{\paperheight}
\setlength{\pdfpagewidth}{\paperwidth}
  
\usepackage{url}
\usepackage{graphicx, color}
\usepackage[font=small,labelfont=bf]{caption}
\usepackage[subrefformat=parens]{subcaption}
\usepackage{amsmath}
\usepackage{subcaption}
\usepackage{algorithm}
\usepackage{multirow}
\usepackage[noend]{algpseudocode}
\usepackage{enumerate}
\usepackage{arydshln}
\usepackage{diagbox}
\usepackage{comment}
\usepackage{amssymb}
\usepackage[utf8]{inputenc}
\usepackage{cleveref}
\usepackage{listings}

%\pagestyle{plain}

\def\eg{\textit{e.g.}}
\def\ie{\textit{i.e.}}
\def\nfactor{\textit{NFActor}}

\begin{document}

\title{\Large \bf NFActor: A Distributed Actor Framework for Building Resillient NFV Systems}

\author{
Paper \#44, xx pages
}

\maketitle

\begin{abstract}

The quick development of Network Function Virtualization (NFV) urges researchers to develop new functionalities for NFV system besides maximizing packet processing capacity. Among these new functionalities, resilience functionalities, such as flow migration and fault tolerance, are hard to tackle and yet very useful in production environment. However, implementing flow migration and fault tolerance requires manually modifying the source code of NF software and providing a control channel for message passing, which may be very tedious to implement and difficult to get right.

In this paper, we present NFActor framework, a framework for building transparently resilient NFV system using actor programming model. NFActor framework provides a set of APIs for constructing NF modules and NF modules written for NFActor framework are transparently resilient. This enables implementers to focus on the core logic design of NF modules without worrying about providing interfaces to implement resilience. Due to the use of actor framework, NFActor provides a very fast migration protocol and a lightweight flow replication protocol.

The evaluation result shows that: First, using NFActor does not incur a significant overhead when processing packet normally and NFActor framework scales well. Second, NFActor out-performs existing works on flow migration by more than 50\% in flow migration completion time. Third, NFActor achieves a consistent recovery time even under increased workload.


\end{abstract}

\section{Introduction}


The recent paradigm of Network Function Virtualization (NFV) advocates moving
Network Functions (NFs) out of dedicated hardware middleboxes and running them as
virtualized applications on commodity servers \cite{nfv-white-paper}. With NFV, network
operators no longer need to maintain complicated and costly hardware middleboxes. Instead,
they may launch virtualized devices (virtual machines or containers) to run NFs on the fly, which
drastically reduces the cost and complexity of deploying network services, usually consisting of a sequence of NFs such as ``firewall$\rightarrow$IDS$\rightarrow$proxy'', {\em i.e.}, a service chain.

 %However, simply running NF software in virtualized environment is not enough to satisfy the stringent requirement of NFV. What network operators need is a full-fledged NFV system, that is capable of handling different kinds of NFV management tasks.
A number of NFV management systems have been designed in recent years, \eg, E2 \cite{palkar2015e2}, OpenBox \cite{OpenBox}, CoMb
\cite{sekar2012design}, xOMB \cite{anderson2012xomb}, Stratos
\cite{gember2012stratos}, OpenNetVM \cite{hwang2015netvm, zhang2016opennetvm}, ClickOS \cite{martins2014clickos}. They implement a
broad range of NF management functionalities, including %line-speed packet processing \chuan{it is not management functionality},
 dynamic NF placement, elastic NF scaling,
load balancing, etc., which facilitate network operators in operating NF service chains in virtualized environments. However, none of the existing systems enable failure
tolerance \cite{rajagopalan2013pico, sherry2015rollback} and flow migration \cite{gember2015opennf, rajagopalan2013split, khalid2016paving} capabilities simultaneously, both of which are of pivotal importance in practical NFV systems for resilience and scalability.
%\cui{add dpdk}

{\em Failure resilience is crucial for stateful NFs.}  Many NFs maintain important per-flow states \cite{EnablingNF}. Intrusion
detection systems such as Bro \cite{bro} parse different network/application
protocols, and store and update protocol-related
states for each flow to alert potential attacks. Firewalls \cite{firewall}
maintain TCP connection-related states by parsing TCP SYN/ACK/FIN packets for
each flow. Some load-balancers \cite{lvs} use a map between flow identifiers and
the server address to modify the destination address in each flow packet.
It is critical to ensure correct recovery of flow states in case of NF instance failures, such that the connections handled by the failed NF instances do not have to be reset. In practice, middlebox vendors
strongly rejected the idea of simply resetting all active connections after failure as it
disrupts users \cite{sherry2015rollback}.

{\em Flow migration is important for long-lived flows in various scaling cases.} Existing NF management systems mostly assume dispatching new flows to newly created NF instances when existing instances are overloaded, or waiting for remaining flows to finish before shutting down a mostly idle instance, which is in fact only feasible in cases of short-lived flows. In real-world Internet systems, long-lived flows are common. Web applications %such as websites and on-line games
 usually multiplex application-level requests and responses in
one TCP connection to improve performance. For example, a web browser uses one TCP connection to exchange many requests and responses with a
web server \cite{http-keep-alive}; video-streaming
\cite{ffmpeg} and file-downloading \cite{ftp} systems maintain long-lived TCP
connection for fetching a large amount of data from CDN servers. %These applications contribute to many long-lived flows.
 When NF instances handling long flows are overloaded, some flows need to be migrated to new NF instances, in order to mitigate overload of the existing ones in a timely manner \cite{gember2015opennf}; when some NF instances are handling a few dangling long flows each, it is also more resource/cost effective to migrate the flows to one NF instance while shutting the others down.
%Without flow migration, the overload of NF instances could not be mitigated in a timely manner because of long flows \cite{gember2015opennf}.


Given the importance of failure resilience and flow migration in an NFV system, why are they absent in the existing NF management systems? The reason is simple: implementing flow migration and fault
tolerance has been a challenging task on the existing NFV software architectures. To provide resilience, important NF states must be correctly extracted from the NF software for transmitting to a new NF instance, needed both for flow migration and replication (for resilience). However, a separation between NF states and core processing logic is not enforced in the state-of-the-art implementation of NF software. Especially, important NF states may be scattered across the code base of the software, making
extracting and serializing NF states a daunting task. Patch codes need to be
manually added to the source code of different NFs to extract and serialize NF
states \cite{gember2015opennf}\cite{rajagopalan2013split}. This usually requires a huge amount of manual work to add up to
thousands of lines of source code for one NF, \eg, Gember-Jacobson {\em et al.}~\cite{gember2015opennf} report
that it needs to add 3.3K LOC for Bro \cite{bro} and 7.8K LOC for Squid caching
proxy \cite{squid}.  Realizing this difficulty, Khalid {\em et al.}~\cite{khalid2016paving} use
static program analysis technique to automate this process. However, applying
static program analysis itself is a challenging task and the inaccuracy of
static program analysis may prevent some important NF states from being
correctly retrieved.

Even if NF states can be correctly acquired and NF replicas created, flows need to be redirected to the new NF instances in cases of NF load balancing and failure recovery. In the existing systems, this is usually handled by a centralized SDN controller, which initiates and coordinates the entire migration process for each flow. Aside from compromised scalability due to the centralized control, for lossless flow migration, the controller has to perform complicated migration protocols that involve multiple passes of messages among the SDN controller, switches, migration source and migration target \cite{gember2015opennf}, which adds delay to flow processing and limits packet processing throughput of the system.
%Even if NF states can be correctly acquired, transmitting
%the states among different NFs %and the NFV system controller
% requires an
%effective message passing service. The existing NF software (\eg,
%Click\cite{kohler2000click}) does not usually provide the support for a messaging channel, and programmers need to manually add this communication
%channel into the NF software. Finally, the additional codes that are patched to
%implement resilience inevitably entangle with the core processing logic of NF
%software. It may lead to serious software bugs if not handled properly.


In this paper, we propose a software framework for building resilient NFV systems, \nfactor, exploiting the actor framework for programming distributed services \cite{actor-wiki, akka, newell2016optimizing}. Our main observation is that actor provides the unique benefits for light-weight, decentralized migration of network flow states, based on which we enable highly efficient flow migration and replication. %We use actors to keep track of the flow states and migrate the actors without a centralized coordinator of a centralized controller.
\nfactor~tracks each flow's state with our high-performance flow actor, whose design transparently separates flow state from NF processing logic. \nfactor~provides service chain processing of flows using flow actors on carefully designed uniform runtime environment, and enables fast flow migration and replication without relying much on centralized control.
%\nfactor~uses actors to process flow packets at a high throughput rate. The flow states are kept by the actors, which can be migrated without a centraliezd coordinator.
%\nfactor~includes the following modules: (i) \ac{a light-weight controller for basic cluster management} (ii) \ac{runtimes that conduct service chain processing using actors} (iii) \ac{virtual switches for balancing the workload on runtimes.} \chuan{briefly introduce key modules in nfactor system}
\nfactor~achieves transparent resilience, easy scalability and high performance in network flow processing based on the following design highlights: %\chuan{improve and add design highlights in the following}

$\triangleright$ {\em Clean separation between NF processing logic and resilience support.} Unlike existing work
\cite{gember2015opennf, sherry2015rollback} that patch functionalities for failure resilience into NF software, \nfactor~provides a clean separation between important NF states and core NF processing logic in each NF using a unique API, which makes extracting, serializing and transmitting important flow states an easy task.
Based on this, the \nfactor~framework can transparently carry out flow migration and replication operations, those needed to enable failure resilience, regardless of the concrete network function to be replicated, \ie, which we refer to as {\em transparent resilience}. Using \nfactor, programmers implementing the NFs only need to focus on the core NF logic, and the framework provides the resilience support. % On the other hand, flow actors can be transparently migrated and replicated as long as they load NF modules that are written with \nfactor~API. We refer to this as transparent resilience in~\nfactor.


$\triangleright$ {\em Per-flow micro-management.} Fundamentally different from the existing systems, \nfactor~creates a micro execution context for each flow by providing a dedicated service chain on one actor for processing packets of this flow on the actor. This can be viewed as a micro (service chain) service dedicated to the flow.
 %Actors in \nfactor~are configured with a rich set of message handlers and run on uniform runtime systems, enabling direct communication between remote actors running on different runtime systems.
%\chuan{idea of the following sentence has been covered by the paragraph above; instead you should describe how this per-flow management can enable better scalability and high speed packet processing}
The micro execution context is constructed using actor framework, which has been proven to be a light-weight and scalable abstraction for building high-performance systems \cite{newell2016optimizing}. Scheduling actors to execute only incurs a small overhead, enabling \nfactor to have a high packet processing throughput. The horizontal scalability of \nfactor is also improved as actors can be scheduled to run on uniform runtime systems.
%Inside this execution context, the flow actor can actively exchange messages with other actors and transmit flow states without disturbing the normal NF processing.


%to provide dedicated service chain services for individual flows,  \chuan{briefly discuss how the design enables easy scalability and high performance in network flow processing}.

%NFs in a service chain are deployed inside the execution context of an actor, instead of being chained through different virtualized devices (\eg, virtual machines or containers). In this way, a unique actor is responsible for processing a network flow through a dedicated service chain. This unique actor fully monitors the flow processing. It can interrupt the flow processing for flow migration or fault tolerance without the need to contact the service chain.


$\triangleright$ {\em Largely decentralized implementation.} Based on decentralized message passing of the actor framework, flow migration and replication in \nfactor~are fully automated, achieved in a fully distributed fashion without continuous monitoring of a centralized controller, which distinguishes \nfactor~from the existing NFV systems \cite{gember2015opennf}. The controller in \nfactor~is only used for controlling dynamic scaling and initiating flow migration and replication,
%\chuan{describe what the controller is used for}
thus light-weighted and failure resilient as the controller does not need to maintain complicated state generated by flow migration and can be easily replicated by storing its simple state on a reliable storage system like ZooKeeper \cite{hunt2010zookeeper}.
%\chuan{clarify why the controller is failure resilient}.
In addition, \nfactor~is implemented on top of the high speed packet I/O library, DPDK \cite{dpdk}, which further improves the performance of \nfactor.

%The actor programming model only imposes a small overhead during service chain processing and flow management, improving the performance of NFActor.


Going beyond resilience, our NFActor framework also enables several interesting applications that the existing NFV systems are difficult to support, including live NF update, flow deduplication and reliable MPTCP subflow processing. These applications require individual NFs to initiate flow migration, %explicitly notify the flow actor to migrate,
which is hard to achieve (without significant overhead) in existing systems where flow migration is initiated and fully monitored by a centralized controller. In \nfactor, these applications can utilize our decentralized and fast flow migration to achieve live NF update with almost no interruption to high-speed packet processing of the NF, best flow deduplication to conserve bandwidth, and correct MPTCP subflow processing, with ease. %NFActor framework also provides live updates to NFs that process packets at the rate of millions packets per second with almost zero down time, due to the blazingly fast flow migration speed.

We implement \nfactor~on a real-world testbed and opensource the project code \cite{projectcode} %\chuan{complete the url in the bib}.
\chuan{improve the result discussion} The result shows that the performance of the runtime system is desirable. The runtimes have almost linear scalbility. The flow migration is blazingly fast. The flow replication is scalable, achieves desirable throughput and recover fast. The dynamic scaling of NFActor framework is good with flow migration. The result of the applications are good and positive.


The rest of the paper is organized as follows. \chuan{to complete} %We present background about NFV and the actor model in Sec.~\ref{sec:background} and overview our \nfactor~system in Sec.~\ref{sec:overview}. We discuss in detail the fault tolerance and flow migration design in Sec.~\ref{sec:fm} and Sec.~\ref{sec:ft}. We show the implementation and evaluation results in Sec.~\ref{sec:implementation} and Sec.~\ref{sec:experiments}, followed by related work in Sec.~\ref{sec:relatedwork}. Sec.~\ref{sec:conclusion} concludes the paper.







%Unfortunately, few existing NFV systems could achieve the goal to resiliently maintain network functions. In most of the existing NFV management systems (i.e. E2 \cite{palkar2015e2}, OpenBox \cite{bremler2015openbox}), network functions have been treated as a black box, which consume packets from ingress ports and generate output packets from egress ports. Therefore these systems only provide a per-NF based management scheme. Even though per-NF based management has been proved to be effective in dealing with dynamic scaling and NF planning, it might compromise the reliability and resilience under certain circumstances. A typical example is during the update to important NF configuration files (i.e. Firewall rule) or to the NF softwares, the NF instances usually need to be shutdown. Due to the limitation of per-NF based management, there is no effective method to prevent established network flows from being forced to shutdown due to this process. The only way towards solving this problem and creating a fully resilient NFV system is to provide efficient per-flow based management, on top of which to achieve flow migration and replication for true resilience.
%In this paper, we propose a new NFV building framework, called NFActor.

%However, several open problems have existed with per-flow based management scheme. It is hard to migrate flows lively without direct support from NF runtime system. Therefore, in per-NF based NFV systems, migrating flows by directly changing the route of the flow may cause serious packet drop and may lead to inconsistent flow states. Several existing works including OpenNF \cite{gember2015opennf} and Split/Merge \cite{rajagopalan2013split} made initial contributions to provide a runtime that supports live flow migration. However, the scalability of their approach is limited by using a centralized controller to coordinate the entire flow migration process. The centralized controller may also be a single point of failure, making their systems vulnerable to software bugs. Also, in order to use their runtime systems, people need to add non-trivial patch code to existing NF softwares, compromising the usability of their systems.

%However, with the developement of NFV, researchers found out that managing at middlebox level could not satisfy the requirement of some applications. Some applications require direct management of a single network flow. A straightforwad example is flow migration. When migrating a flow, the NFV management system must transfer the state information associated with the flow from one middlebox to another, and redirecting the flow to the new middlebox in the mean time. Another example is fault tolerance of an individual flow. The NFV management system has to replicate flow's state on a replica and recovers flow's state on a new middlebox in case of the failure of the old middlebox.

%Realizing these limitations, we propose a new NFV management system in this paper, called NFActor. NFActor framework is designed to provide transparent, scalable and efficient flow management. NFActor framework achieves this goal by creating a micro execution context, running on top of a uniform runtime system, for each flow using actor programming model. This execution context is augmented with several message handlers to achieve basic service chain processing and flow management tasks. To transparently integrate NF softwares with the execution context, we provide a new programming interface for creating new NF modules for NFActor framework, which enforces separation between the core NF processing logic and the NF state of each flow.

%Using NFActor framework, programmers could concentrate on the internal logic design of the NF. NFs written using the NFActor programming interface could be transparently integrated with the flow management tasks provided by the actor execution context. Due to the message passing and decentralized nature of actor programming model, the flow management tasks are fully automated by actor scheduling and message passing. There is no need for the continuous monitoring of a centralized controller. Therefore the controller in NFActor framework is extremely light-weighted and failure resilient. The actor programming model only imposes a small overhead during service chain processing and flow management, improving the performance of NFActor.

%Besides strong resilience through per-flow management, NFActor framework also enables several interesting new applications that existing NFV systems are hard to provide. These applications utilize the feature of decentralized flow migration to reduce the output bandwidth consumption during deduplication and ensures correct MPTCP subflow processing. NFActor framework also provides live updates to NFs that process packets at the rate of millions packets per second with almost zero down time, due to the blazingly fast flow migration speed.

%We implement NFActor framework on top of DPDK and evaluate its performance. The result shows that the performance of the runtime system is desirable. The runtimes have almost linear scalbility. The flow migration is blazingly fast. The flow replication is scalable, achieves desirable throughput and recover fast. The dynamic scaling of NFActor framework is good with flow migration. The result of the applications are good and positive.

\section{Background}
\label{sec:background}

\subsection{Network Function Virtualization}

A NFV system \cite{nfv-white-paper} typically consists of a controller and
many VNF instances. Each VNF instance is a virtualized device running NF software. VNF instances are connected into service chains, implementing certain network services, \eg, access service. Packets of a network flow go through the NF instances in a service chain in order before reaching the destination.

A VNF instance constantly polls a network interface card (NIC) for packets. Using traditional kernel network stack incurs high context switching overhead \cite{martins2014clickos} and greatly compromise the packet processing throughput. To speed things up, hypervisors usually map the memory holding packet buffers directly into the address space of the VNF instances with the help of Intel DPDK\cite{dpdk} or netmap \cite{netmap}. VNF instances then directly fetch packets from the mapped memory area, avoiding expensive context switches. Recent NFV systems \cite{palkar2015e2, Han:EECS-2015-155, sherry2015rollback, martins2014clickos, hwang2015netvm} are all built using similar techniques.

%尽管使用DPDK或netmap来实现高速的包处理已经成为了一种必然的趋势,现有的流管理系统仍然在使用kernel networking stack来实现communication channel。NFActor则完全抛弃了kernel networking stack。NFActor利用DPDK构建了自己的可靠传输系统。这套可靠传输系统可实现6M/s的消息传输吞吐量。更重要的是,使用这套可靠传输系统不会引发任何的context switching,这进一步提高了整个系统的速度。 我们利用一个中央化的调度器将可靠传输模块与其他模块无缝的衔接起来

Even though using DPDK and netmap to improve the performance of packet processing has become a new trend. Existing flow management systems are still using kernel networking stack to implement the communication channel. On contrary, NFActor completely abandons the kernel networking stack, by constructing a reliable transmission module using DPDK. Using this reliable transmission module does not incur any context switches, thereby boosting the message throughput to 6 million messages per second in our evalution.

\subsection{Actor Model}

The actor programming model has been used as the basic building block for constructing massive, distributed systems\cite{actor-wiki, akka, newell2016optimizing}. Each actor is an independent execution unit, which can be viewed as a logical thread. In the simplest form, an actor contains an internal actor state (\eg, statistic counter, status of peer actors), a mailbox for accepting incoming messages and several message handler functions. An actor can process incoming messages using its message handlers, send messages to other actors through the built-in message passing channel, and create new actors.

There are several popular actor frameworks, \ie, Scala Akka \cite{akka}, Erlang \cite{erlang}, Orleans \cite{Orleans} and C++ Actor Framework \cite{caf}. These actor frameworks have been used to build a broad range of distributed programs, including on-line games and e-commerce. For example, Blizzard (a famous PC game producer) and Groupon/Amazon/eBay (famous e-commerce websites) all use Akka in their production environment \cite{akka}.

%actor model可以被很自然的拿来构建flow的执行环境。在一个VNF instance里,我们为每一个flow都创建一个actor. 并将flow包的处理对应的actor的消息处理上。与此同时,其他的流管理功能可以以消息处理函数的形式被添加到流actor上。%但是,已有的actor的系统都没有基于NFV的执行环境进行优化。在我们最初版本使用了libcaf作为actor执行库,但是却无法得到一个令人满意的结果。因此我们制作了自己的actor执行模型并大幅度提升了性能。

Actor model is a natural fit when buildling flow execution context. In a VNF instance, we can create one actor for one flow, and map the flow packet processing to actor message processing. In the mean time, the flow management tasks could be implemented as message handlers on the actor. However, none of the existing actor systems are optimzed for NFV envirnoment. In our initial prototype, we use C++ Actor Framework \cite{caf} to build NFActor, but the performance of that prototype turns out to be not satisfactory. This forces us to make a customized actor model for NFActor and greatly improves the performance. 

%If we treat a NF instance as an actor, then the incoming packets could be viewed as messages that are sent to this actor's mailbox. Processing packets could be mapped to handling messages using NF software's message handler. Even though there is a simple and clear relationship between NF instances and actor model, there has been no attempt to build NFV system using actor model.

\section{Design}

%以下两图展示了nfactor系统的基本架构。nfactor框架将统一的运行时系统组成cluster。在这个nfactor cluster内,细粒度的流管理可以被快速的执行。这个cluster由一个轻量级的控制器进行控制以实现动态扩展,和开启流迁移以及容错。

Figure \ref{fig:runtime} and \ref{fig:runtime-arch} daemonstrate the basic architecture of NFActor framework. NFActor framework composes uniform runtime systems into a cluster. Within this cluster, fine-grained flow managmeent tasks could be quickly executed. This cluster is controlled by a lightweight controller for dynamic scaling and inititiating flow management tasks, including flow migration and fault tolerance.

%nfactor的设计遵循了以下几个原则。
The design of NFActor framework follows the following principles.

\begin{itemize}

%第一,低开销。在提供复杂流管理的同时,nfactor也必须高速的处理数据包。因此,nfactor为每一个流所提供的execution context必须是一个轻量级的抽象,它不能对正常的NF处理产生比较大的性能影响。The execution context of nfactor framework是利用actor programming model 进行构建的。我们实现了自己的actor programming model库,从编程的角度考虑,消息传递的开销仅相当于一次函数调用。

\item \textbf{Low Overhead.} While providing compliated flow management tasks, the runtime system of NFActor framework must be able to process packets at high speed. Therefore, the execution context that NFActor created for each flow must be a lightweight abstraction, it should not compromise the processing speed of a NF. In NFActor, this execution context is constructed using actor programming model. We implement our own actor programming model to minimize the overhead associated with the execution context.

%第二,效率。 为了适应高速nfv系统的需求,nfactor所提供的流管理必须十分高效。在高速nfv系统中,一个NF每秒钟需要处理几百万个包,并控制几万流。在如此高速的吞吐量下实现流管理功能,这就要求nfactor避免使用内核网络协议栈,以避免上下文切换的的开销。在nfactor系统中,所有的数据处理,无论是dataplane的数据包还是actor发送的远程消息,都是利用高速的包IO (DPDK)来实现的。

\item \textbf{Efficiency.} To accomodate the need of high speed NFV systems, the flow management tasks of NFActor must be highlighy efficient. In high speed NFV systems, a NF may process millions of packets every second and handle tens of thousands of flows. To achieve high efficiency, NFActor completely abandoned using kernel networking stack to avoid the overhead of context switching. In NFActor, all the data, whethere it's data plane packet or remote actor messages, are transmitted through high-speed packet I/O (i.e. DPDK).

%第三,可扩展性。现代nfv系统必须具有良好的可扩展性,以适应不断变化的网络流量的需求。为了实现良好的可扩展性,nfactor系统提供了统一的运行时系统,并使用多个运行时系统组成nfactor cluster。在nfactor cluster内,运行时系统可以对经过自己的流进行路由管理,以实现动态的负载均衡。同时,我们也简单而快速的在不同的runtime之间传递信息,以实现高效的流管理任务。

\item \textbf{Scalability.} Modern NFV system must have good scalability, to accomodate varying network traffic. To provide good scalability, NFActor framework provides uniforms runtimes and connects multiple runtimes into a NFActor cluster. Within this cluster, the runtime system could manage output route for each flow that passes through it, to achieve dynamic load balancing. The uniform runtime design also faciliatates message passing among different runtimes, to achieve efficient flow management tasks.

\end{itemize}

\subsection{Runtime Cluster}


\begin{figure}[!t]
\begin{subfigure}[t]{0.30\linewidth}
   \centering
   \includegraphics[width=\columnwidth]{figure/nfactor-runtime-with-port.pdf}
   \caption{A runtime with three ports.}\label{fig:runtime-with-port}
  \end{subfigure}\hfill
  \begin{subfigure}[t]{0.69\linewidth}
 \centering
   \includegraphics[width=\columnwidth]{figure/nfactor-runtime-connection.pdf}
   \caption{A minimal runtime connection.}\label{fig:runtime-with-io-runtime} \end{subfigure}\hfill
   \begin{subfigure}[t]{0.99\linewidth}
  \centering
    \includegraphics[width=\columnwidth]{figure/nfactor-cluster.pdf}
    \caption{A NFActor runtime cluster consists of multiple lay, controlled by a controller.}\label{fig:runtime-cluster} \end{subfigure}
 \caption{The flow migration performance of \nfactor}
\label{fig:runtime}
\end{figure}

%图1c展示了运行时cluster的基本结构。运行时cluster是由若干层运行时连接而成,并由一个轻量级的控制器利用RPC进行管理。
Figure \ref{fig:runtime-cluster} gives an example of a running NFActor cluster. The NFActor cluster consists of multiple runtimes controlled by a light-weight controller through RPC.

%图1a给出了一个runtime的基本结构。一个runtime包含三个端口。输入端口和输出端口用来接收和传递数据层的数据包。控制端口专门用来传递flow actor执行流管理任务时所生成的远程消息。输入输出端口也可以用来传递远程消息,我们在后面的章节中给出详细的解释。
Figure \ref{fig:runtime-with-port} shows the basic struture of a runtime. A runtime consists of three ports. The input and output ports are used to receive and send dataplane packets. The control port is only used to transmit remote messages when flow actor executes flow management tasks. Both input and output ports could also be used to transmit remote messages, this is further illustrated in later chapters.

%一个运行时系统自己并不能执行负载均衡以及流管理任务,它需要和其他的运行时系统进行连接才能完成这些功能。每一个运行时系统的三个端口都可以和其他多个运行是系统进行连接。图1b给出了一个能实现各种流管理功能的最小连接图。在图1b中,2号运行时系统的输入端口和输出端口分别与1号运行时系统的输出端口和4号运行时系统的输入端口相连,因此对于2号运行时系统而言,1号运行时系统时它的input runtime, 4号运行时系统时它的output runtime.同理,对于1号运行时系统而言,2号运行时系统时它的output runtime. 在NFActor framework中,运行时系统可以将自己的输出流量均匀分布在所有的输出运行时系统中。因此我们可以看到 dataplane traffic可以从图1b连接的一端进入并从另一端输出。

A runtime system could not execute any load-balacning and flow management tasks by itself. It needs to be connected with other runtimes and collaborate with those runtimes. The three ports of a single runtime system could be connected to multiple runtimes. Figure \ref{fig:runtime-with-io-runtime} gives a minimal runtime connection that is able to achieve load-balancing and flow management tasks. In figure \ref{fig:runtime-with-io-runtime}, the input and output ports of runtime 2 and 3 are connected with the output port of runtime 1 and input port of runtime 4. From the perspective of runtime 2, runtime 1 is its input runtime and runtime 4 is its output runtime. Similarly, runtime 2 is the output runtime of runtime 1. In NFActor framework, a runtime could balance its workload among all of its output runtimes. This is why the dataplane traffic could enter from one end of the connection in figure \ref{fig:runtime-with-io-runtime} and exit from the other end.

%对与2号和3号运行时系统而言,他们具有相同的input runtime和output runtime。我们将这一类运行时系统归纳入通一个layer。同一个layer的运行时系统的控制端口会被相互连接起来。同时,同一个layer的运行时系统之间可以实现快速高效的流迁移和容错。

From the perspective of runtime 2 and 3, they share the same input runtimes and output runtimes. These runtimes are classified into the same layer. The control ports of runtimes under the same layer are directly connected, so that flows could be quickly migrated and replicated among runtimes under the same layer.

%如图2所示,我们可以构建一个由multipile layers of runtime 所组成的nfactor cluster。这个cluster由一个轻量级的controller进行控制。controller可以检测每一个runtime的workload以实现动态扩展。同时,controller可以通过rpc来发起流管理任务的执行。

As shown in figure \ref{fig:runtime-cluster}, we can construct a NFActor cluster by creating multiple layers of runtimes. This runtime cluster could be controlled by a controller, which monitors the workload on each runtime for dynamic scaling. This controller could also initiate flow management tasks by sending RPC requests to selected runtimes. 

\begin{figure}
		\centering
		\includegraphics[width=\columnwidth]{figure/nfactor-runtime-arch.pdf}

		\caption{The internal architecture of a NFActor runtime system. }
\label{fig:runtime-arch}
\end{figure}

%\section{Resilience}

In this section, we present the fault tolerance mechanism and flow migration protocol used by NFActor framework. Before going into the details, we first compare the difference between 

\subsection{Fault Tolerance}
\label{sec:ft}

In this section, we introduce the fault tolerance mechanisms for our NFActor
framework, including the controller, the virtual switch, and the runtimes. 
Depending on the nature of these three components, we carefully design 
lightweight fault-tolerance mechanisms for them so that these mechanisms be 
robust and have little performance impact in normal case.

\subsubsection{Replicating Controller}

Since the controller is a singled-threaded server that stores the states of 
the NFActor runtime, we persistently log these states and replicate them. The 
controller only needs to log the state of each NFActor runtime in 
the cluster view list. Whenever the controller needs to modify the state of a 
runtime, it logs the intended operation, modifies the state and logs a success 
mark for the intended operation.

The liveness of the controller is monitored by a guard process and the 
controller is restarted immediately in case of failure. On a reboot, the 
controller reconstructs the state in the cluster view list by replaying log. 
Each runtime in the cluster monitors the connection status with the controller 
and reconnects to the controller in case of a connection failure.

\subsubsection{Replicating Virtual Switch}

The most important state of the virtual switch process is its switching hash 
table in memory. In order to replicate the virtual switch, we constantly 
check-point the container memory image of the virtual switch using CRIU 
\cite{criu}, a popular tool for checkpoint/restore Linux processes. One main 
technical challenge is that CRIU has to stop a process before checkpointing it, 
which may hurt the availability of the virtual switch.

We tackle this challenge by letting the virtual switch call a 
fork() periodically (by default, one minute), and then we use CRIU to checkpoint 
the child process. Therefore, the virtual switch can proceed without affecting 
NFActor performance.

% Since the
% check-pointing needs to halt the execution of the whole virtual switch, it can
% not be frequently performed. In our implementation, we create a check-point for
% every second. However, this means that some states in the switching hash table
% might be lost and the flow connection related with these states may be forced to
% terminate. We argue that this is an acceptable implementation trade-off as flows
% could do a reconnection to complete unfinished tasks. If strong consistency is
% required, the virtual switch could use the replication strategy of
% FTMB\cite{sherry2015rollback}, we leave this to the future work.

\subsubsection{Replicating Runtime}

\begin{figure}
	\begin{subfigure}[b]{0.45\columnwidth}
		\centering
		\includegraphics[width=\columnwidth]{figure/NFActor-Runtime-Replicate.pdf}
		\caption{Actor replicates its flow state to another
		runtime.}\label{fig:replicate} \end{subfigure}\hfill
	\begin{subfigure}[b]{0.45\columnwidth}
		\centering
		\includegraphics[width=\columnwidth]{figure/NFActor-Runtime-Recover.pdf}
		\caption{A failed runtime is restarted.}\label{fig:recover}
	\end{subfigure}%
\caption{Replication strategy for NFActor runtime.}
\end{figure}

To perform lightweight NFActor runtime replication, we leverage the actor 
abstraction and state separation to create a lightweight flow state replication 
strategy. In NFActor runtime, important flow states associated with a flow is 
owned by a unique actor. NFActor runtime can replicate each actor 
independently without incurring the overhead of check-pointing the entire 
container images \cite{sherry2015rollback, rajagopalan2013pico}.

This primary-backup replication manner can tolerate one actor failure at 
runtime, and we think this fault-tolerance guarantee is sufficient because the 
chance two actor machines failing at the same time is extremely rare.

\textbf{Find A Replication Target}: When the actor is created, it selects a
\textit{running} state runtime with the smallest workload as its replication
target. Then the actor negotiates with the replication target about whether the
replication target can accept this actor's replica. In case that replication
target refuses to store the actor's replica, the actor tries to select
another replication target. 

\textbf{Flow State Replication (figure \ref{fig:replicate})}: After determining
the replication target, the actor performs flow state replication. Before the input packet is processed on
the service chain, the actor saves a copy of the input packet. After the packet
finishes being processed on the service chain, the actor sends the local runtime
ID, flow identifier, original input packet and the processed packet to the
replication target. For every fixed number of packets that the actor has processed, the actor create a
snapshot of all the flow states and send the snapshot to the replication target as well.

When the replication target receives a new replication message, it saves the
message content in the RAM. If received message includes a new state snapshot,
replication target wipes out all previously saved content associated with the flow and save the new
state snapshot as well. Then the replication target sends the processed
packet out from its output port to the virtual switch. The flow state
replication procedure ensures the same output commit property as indicated in
\cite{sherry2015rollback}.

\textbf{Recover Failed Runtime (figure \ref{fig:recover})}: If a NFActor runtime
fails, it's failure will be detected by the controller after a timeout. Then the failed NFActor runtime will be restarted
with the same runtime ID. The restarted NFActor then starts recovery process. It
sends to all the other NFActor runtimes in the cluster a recover message, which
contain its NFActor runtime id. Other NFActor runtimes respond to the recover
message by sending all the replicas with the same runtime ID back to the
restarted NFActor runtime. The restart NFActor runtime then use these replicas
to reconstruct its state before failure. When the NFActor runtime finishes
recovery, it sends a {\tt join} message back to the controller to re-join the
cluster. 

% \textbf{Discussion}: The runtime replication strategy can tolerate at most one
% failed runtime. If multiple runtimes fail concurrently, then the replicas will
% be lost and the states of some flows will never be correctly recovered.
% Multi-machine replication algorithm such as PAXOS \cite{chandra2007paxos} could
% be used to replicate the flow state to multiple backups. However, PAXOS may not
% be efficient enough to handle the replication of a large number of packets. We
% leave exploring multi-backup to our future work. 

%%%%%%%%%%%%%%%%%%%%%%%%%%%The following is flow migration%%%%%%%%%%%%%%%%%%%%%%%%%

\subsection{Distributed Flow Migration}
\label{sec:fm}

In this section, we present the distributed flow migration method of NFActor
framework. There are some problems with existing flow migration
framework \cite{gember2015opennf, rajagopalan2013split}. First they all use a
centralized controlelr to monitor the entire flow migration
process. This is not scalable. The second problem is that the flow migration
protocol is too complicated to implement. Patch code needs to be added to the
core processing logic of the NFs. The flow migration protocol needs to exchange
multiple messages among controller, switches and NF instances. This increases
the possibility of serious software bugs. The final problem is that the NF can
not start the flow migration process by itself. It must be started from the
controller. This stops the NFs from making timely responses to overload signal.

All the above mentioned problems could be solved using NFActor framework. In
NFActor framework, the NF packet processing is carried out within the
execution context of the actor. This additional layer of monitoring gives
NFActor power to start flow migration from the NFActor runtime. On the other
hand, the API design forces a clean separation between flow state and NF
processing logic, making extracting and serializing flow state an easy task.
NFActor framework also has a built-in message-passing functionality that
makes flow migration simpler. In this section, we elaborate in detail how flow
migration works in NFActor framework.

\subsubsection{Distributed Flow Migration Protocol}

\begin{figure*}[!t]
	\begin{subfigure}[t]{0.23\linewidth}
		\centering
		\includegraphics[width=\columnwidth]{figure/NFActor-Flow-Migration-Init.pdf}
		\caption{Initiate flow migration.}\label{fig:init} 
	\end{subfigure}\hfill
	\begin{subfigure}[t]{0.23\linewidth}
		\centering
		\includegraphics[width=\columnwidth]{figure/NFActor-Flow-Migration-First.pdf}
		\caption{Create target actor.}\label{fig:first} \end{subfigure}\hfill
	\begin{subfigure}[t]{0.23\linewidth}
		\centering
		\includegraphics[width=\columnwidth]{figure/NFActor-Flow-Migration-Second.pdf}
		\caption{Contact virtual switch.}\label{fig:second}
	 \end{subfigure}\hfill
	 \begin{subfigure}[t]{0.23\linewidth}
		\centering
		\includegraphics[width=\columnwidth]{figure/NFActor-Flow-Migration-Third.pdf}
		\caption{Migrate flow state.}\label{fig:third}
	 \end{subfigure}
\caption{Distributed Flow Migration Protocol of NFActor Framework}
\label{fig:migration}
\end{figure*}

The distributed flow migration protocol is shown in figure \ref{fig:migration}.
It consists of passing 4 request-response. We first show successful
request-response in figure \ref{fig:migration} and then supplement possible
failure cases.

\textbf{Initiate Flow Migration}: As shown in figure \ref{fig:init}, the flow
migration in NFActor is initiatied by sending a {\tt start\_migration} message
from the NFActor runtime 1 to an actor \textit{a} on this runtime. The {\tt
start\_migration} message contains a ID of a target runtime. Runtime 1 acquires
this ID by querying the view service. Once actor \textit{a} receives this
message, it starts the migration process by itself, without involving a centralized controller.

Runtime 1 is promised to receive a response from actor \textit{a}. A {\tt
migration\_ok} message sent from actor \textit{a} indicates that the migration
is successfully finished. The flow being migrated has
resumed its execution on the target runtime. Actor \textit{a} has quit
execution and released all its resources. The migration can of course fail due
to many reasons. In that case, actor \textit{a} will respond a {\tt
migration\_fail} message back to runtime 1 and continue to process flow packets
on runtime 1.

\textbf{Create Target Actor:} Actor \textit{a} then sends a {\tt create\_target\_actor} request message to the
target runtime 2, expecting a response. This message also contains the flow
identifier of actor \textit{a}. Target runtime 2 target runtime 2 creates a target
actor \textit{b}, registers target actor \textit{b} using the flow identifer
contained in the {\tt start\_migration} message and delegates the migration
process to target actor \textit{b}. Target actor \textit{b} then responds a {\tt
ok} message to actor \textit{a}.

\textbf{Contact Virtual Switch:} Actor \textit{a} sends a {\tt
forward\_to\_target\_runtime} request message to the virtual switch. This
message contains the flow identifier and the ID of the target runtime 2. After
virtual switch has received the request message, the virtual switch first
updates its switching hash table by changing the value associated with the flow identifier
to the ID of target runtime 2. Then the virtual switch creates a data-plane
packet with the same flow identifier contained in the message and fills the
payload with a global unique magic-number. Then the virtual switch sends this
data-plane packet back to runtime 1.

When actor \textit{a} receives the data-plane packet with the magic number, it
knows that the no more flow packets will be forwarded to itself. Then actor
\textit{a} can safely migrate its flow state to the target actor \textit{b}. The
use of the data-plane packet instead of control message ensures lossless flow
migration \cite{gember2015opennf}. In case that actor \textit{a} fails to
receive data-plane response packet after a timeout, actor \textit{a} will
re-send {\tt forward\_to\_target\_runtime} to the virtual switch.

After virtual switch updates its switching hash table, target actor \textit{b}
starts to receive flow packets. Target actor \textit{b} only buffers these
packets and waits until flow state migration complete before processing them.

\textbf{Migrate Flow State:} Actor \textit{a} sends a {\tt migrate\_flow\_state}
request message, along with the serialized flow states of actor \textit{a}, to
target actor \textit{b}. After receving the request message, target actor
\textit{b} first responds a {\tt ok} message back to actor \textit{a}.
Then it drains all the buffered packets and resumes normal flow packet
processing. 

Actor \textit{a} on the other hand, waits until it receives {\tt ok} message
from target actor \textit{b}. Then it responds a {\tt migration\_ok} message
back to runtime 1, destroies all the of its resources and quits.

\subsubsection{Failure Handling}
 
The flow migration protocol in figure \ref{fig:first} to \ref{fig:third} may
generate different kinds of failures. This causes actor \textit{a} in figure
\ref{fig:init} to responds a {\tt migration\_fail} message back to runtime 1. In
this section, we analyze possible failures and how to handle these failures.
Flow migration protocol relies on failure recovery to handle some serious
failures. For the flow migration process, the target actor does not maintain a
replica. In the meantime, the actor being migrated does not log that it is being
migrated. This means that if the runtime involved in the migration fails, the
migration is implicitly stopped.

\textbf{Target runtime is overloaded.} In figure \ref{fig:first},
when target runtime 2 receives {\tt create\_target\_actor} message, it first
checks whether it is overloaded before creating target actor \textit{b}.
This is because runtime 1 uses the view service to determine target runtime and
the view service is not an always-up-to-date service. Target runtime 1 may
experience overload and it can not accept new migrated actors. In that case,
target runtime 2 respond a {\tt fail} response back to actor \textit{a}, causing
actor \textit{a} to stop the migration.

\textbf{Messages are lost in the network.} In figure \ref{fig:first}, if either
the {\tt create\_target\_actor} message or the response message is lost, actor
\textit{a} stops migration. This also causes a timeout to be triggered on target
actor \textit{b} and stops \textit{b}. Figure \ref{fig:second} and
\ref{fig:third} involve changing packet forwading path and migrating flow state.
The migration can not be simply stopped due to message loss. In figure
\ref{fig:second}, actor \textit{a} just keeps re-sending the request message
until it receives a definitive response. In figure \ref{fig:third}, actor
\textit{a} keep retrying for a fixed number of times. Then it performs a live
recovery.

\textbf{Runtime fails.} In figure \ref{fig:first}, the failure of either runtime
simply halts the migration process. In figure \ref{fig:second} and
\ref{fig:third}, actor \textit{a} and target actor \textit{b} monitor each
other. The failure of runtime 1 causes target actor \textit{b} not to receive
{\tt migrate\_flow\_state} message. Actor \textit{b} requests the virtual switch
to forward to runtime 1. The failure of target runtime 2 causes actor \textit{a}
to perform a live recovery. 

\textbf{Buffer Overloads.} After virtual switch updates its switching hash
table, target actor \textit{b} starts to buffer the packets. We set a maximum
capacity for the buffer. If the buffer is full when receiving {\tt
migrate\_flow\_state} message, target actor \textit{b} responds a {\tt fail}
message to actor \textit{a}, causing \textit{a} to perform a live recovery.

\textbf{Virtual Switch Fails.} After virtual switch is restarted, it may lost
some states in the switching hash table. The flow being migrated may be forced
to terminate due to inconsistent state.










\section{Implementation}
\label{sec:implementation}

%我们使用C++来实现NFActor。在当前的实现中,NFActor的每一个runtime都跑在一个container中,不同runtime通过BESS进行连接。每一个runtime的端口都是一个BESS的zerocopy vport。

%重用bess module system。
%NFActor runtime的内部架构复用了BESS的module系统以及bess的调度器。我们将bess module system和scheduler代码port到了nfactor runtime的代码里。并且用

NFActor framework is implemented in C++. The core functionality of NFActor framework contains around 8500 lines of code. We use BESS \cite{bess}\cite{2015} as the dataplane inter-connection tool to connect different runtimes and virtual switches. The three ports that are assigned to each runtime are zero-copy VPort in BESS, which is a high-speed virtual port for transmitting raw packets. BESS could build a virtual L2 ethernet inside a server and connect this virtual ethernet to the physical L2 ethernet. By connecting the virtual L2 ethernet with the ports of runtimes, We can connect different runtimes running on different servers together.

\subsection {Reuse BEES Module System}

The runtime needs to poll packets from the input port, schedule flow actors to run and transmit remote actor messages. To coordinate these tasks, we decide to reuse BESS module systems. BESS module system is specifically designed to schedule packet processing pipelines in high-speed NFV systems, which is a perfect suit to NFActor runtime architecture. We port the BESS module system and BESS module scheduler to the runtime and implement all the actor processing tasks as BESS modules. These modules are connected into the following 5 pipelines.

\begin{itemize}

\item The first/second pipeline polls packets from the input/output port, runs actor scheduler on these packets and sends the packets out from the output/input port.

\item The third pipeline polls packets from control ports, reconstruct packet stream into remote actor messages and send the actor messages to the receiver actors. (The first/second pipeline also carries out this processing because remote messages are also sent to input/output port \ref{}).

\item The fourth pipeline schedules coordinator actor to execute RPC requests sent from the controller. In particular, coordinator actor updates the configuration information of other runtimes in the cluster and dispatches flow migration initiation messages to active flow actors in the runtime.

\item When processing the previous four pipelines, the actors may send remote actor messages. These messages are placed into ring buffers \ref{}. The fifth pipeline fetches remote actor messages from these ring buffers and sends remote actor messages out from corresponding ports.

\end{itemize}

The runtime uses BESS scheduler to schedule these 5 pipelines in a round-rubin manner to simulate a time-sharing scheduling.

\subsection{Customized Actor Library}

To minimize the overhead of actor programming, we implement our own actor library. Due to the single-worker-thread design, when actor transmits local messages, there is no need to use a mailbox \cite{caf} \cite{akka} protected by synchronization primitives to receive the message. The local message transmission are directly implemented as a function call, therefore eliminate the overhead of enqueuing and dequeuing the message from the mailbox. For remote actor message passing, we assign a unique ID to each runtime and each actor. The sender actor only needs to specify the receiver actor's ID and runtime ID, then the reliable transmission module \ref{} could deliver the remote actor message to the receiver actor.

To schedule flow actors, we directly run a flow actor scheduler in the first three pipelines. The flow actor scheduler is able to access the high-speed hash maps for storing flow-key to actor mapping and actor-id to actor mapping in the flow classifier \ref{}. The flow actor scheduler directly indexes the hash map using the key contained in the incoming actor messages and redirect the message to the actor. The coordinator actor is scheduled by the fourth pipeline. The coordinator actor also has accesses to the hash maps in the flow classifier, therefore it is able to forward messages to other flow actors in the runtime.

This simple actor programming could not achieve perfect message matching and complete separation of the internal actor state, as other mature actor frameworks do \cite{akka} \cite{caf}. However, due to its simple architecture, it only imposes a small overhead when doing actor processing, therefore it is able to satisfy the high-speed packet processing requirement of modern NFV system.

\subsection{Reliable Message Passing Module}

To reliably deliver remote actor messages, we build a customized reliable message passing module for NFActor framework. Unlike user-level TCP stack, where messages are inserted into a reliable byte stream and transmitted to the other end, the reliable message passing encodes messages into reliable packet streams.

The reliable message passing module creates one ring buffer for each remote runtime. When an actor sends a remote message, the reliable transmission module allocates a packet, copy the content of the message into the packet and then enqueue the packet into the ring buffer. A message may be splitted into several packets and different messages do not share packets. When the fifth pipeline is scheduled to run, the packets containing remote messages are dequeued from the ring buffer. These packets are configured with a sequential number and sent to their corresponding remote runtimes. The remote runtime sends back acknowledgement packets. Retransmission is fired up in case that the acknowledgement for a packet is not received after a configurable timeout (10 times of the RTT).

We do not use user-level TCP \cite{mtcp} to implement the reliable message passing module. Because compared with our simple goal of reliably transmitting remote actor messages over an inter-connected L2 network, using a user-level TCP imposes too much overhead for reconstructing byte stream into messages. The packet-based reliable message passing provides additional benefits during flow management tasks. For instance, because the second response in the flow migration protocol is sent as a packet on the same path with the dataplane flow packet, it enables us to implement lossless migration with ease. Also, during flow replication, we can directly send the output packet as a message to the replica, without the need to do additional packet copy. 

\section{Evaluation}
\label{sec:experiments}

%Evaluate \nfactor system.

%\chuan{remember to show how many lines of code or other overhead needed for implementing NFV anew on the actor framework}

We evaluate \nfactor framework using three Dell R430 Linux servers, containing 20 logical cores, 48GB memory and 2 Intel X710 10Gb NIC. The Dell servers are connected through a 10GB switch. We use one server to run traffic generators and six virtual switches, which are capable of generating 64 byte data packets at almost 14Mpps, achieve line-rate throughput. We use the reset of the two servers to run runtimes.

To evaluate the performance of \nfactor, we implement 3 customized NF modules using the API provided by \nfactor~framework, the 3 NF modules are flow monitor, firewall and HTTP parser. The flow monitor updates an internal counter when it receives a packet. The firewall maintains several firewall rules and checks each received packet against the rule. If the packet matches the rule, a tag in the flow state is flipped and later packets are automatically dropped. The firewall also records the connection status of a flow in the flow state. For the HTTP parser, it parses the received packets for the HTTP request and responses. The requests, responses and the HTTP method are saved in the flow state. %Throughout the evaluation, we use a service chain consisting of  ``flow monitor$\rightarrow$firewall$\rightarrow$http parser'' as the service chain. We generate evaluation traffic using the BESS's FlowGen module and we directly connect the FlowGen module to the external input port of the virtual switch.

%Table goes  here to explain the functionalities of the NF modules.
The rest of the section tries to answer the following questions. \textit{First, } what is packet processing throughput of~\nfactor, does it scales well? \textit{Second,} how good is the flow migration performance of~\nfactor when compared with existing works like OpenNF? \textit{Third,} how is the performance of flow replication? \textit{Fourth,} how good is~\nfactor's dynamic scaling algorithm performs. The section concludes with the performance of the unique applications that are enabled through~\nfactor's distributed flow migration.

%how well can~\nfactor~scales, in terms of the number of runtimes running inside the system? (Sec.~\ref{sec:normal})

%what is the packet processing capacity of \nfactor~framework? (Sec. \ref{sec:ppc}) \textit{Second, } how well is \nfactor~scales, both in terms of the number of worker threads used by a runtime and the number of runtimes running inside the system? (Sec. \ref{sec:ppc}) \textit{Third, } how good is the flow migration performance of \nfactor~framework when compared with existing works like OpenNF? (Sec. \ref{sec:fmp}) \textit{Fourth, } what is the performance overhead of flow state replication and does the replication scale well? (Sec. \ref{sec:rp})

\subsection{Packet Processing Throughput}
\label{sec:ppc}

\begin{figure}[!t]
	\centering
	\includegraphics[width=\columnwidth]{figure/revised-throughput-test.pdf}
	\caption{The packet processing throughput using different number of runtimes. The runtimes are configured with different NFs and different service chains.
  \textbf{FM}: Flow Monitor. \textbf{HP}: HTTP Parser. \textbf{FW}: Firewall.}
\label{fig:normal-case-eval}
\end{figure}


 %\begin{figure}[!t]
%	\begin{subfigure}[t]{0.49\linewidth}
%		\centering
%		\includegraphics[width=\columnwidth]{figure/nf_throughput_evaluation.pdf}
%		\caption{Packet processing capacity of a single \nfactor~runtime system running with different number of worker threads.}\label{fig:normal-performance} \end{subfigure}\hfill
%	 \begin{subfigure}[t]{0.49\linewidth}
%		\centering
%		\includegraphics[width=\columnwidth]{figure/runtime_pktthroughput.pdf}
%		\caption{Aggregate packet processing capacity of several \nfactor~runtimes.}\label{fig:scalability-performance}
%	 \end{subfigure}
%\caption{The performance and scalability of \nfactor~runtime, without enabling flow migration }
%\label{fig:performance}
%\end{figure}

Figure \ref{fig:normal-case-eval} illustrates the packet processing throughput using differnt number of runtimes. The traffic generators generates flows each lasts for 10 second with a 20 pps (packet per second) flow rate. The packet size is 64 byte. We increase the number of concurrently generated flows to generate packets at around 14Mpps, reaching NIC line-rate. We set up different number of runtimes and configure runtimes with different service chains and calculate the cumulative packet rate from all the runtimes. From \ref{fig:normal-case-eval}, we can see that the runtime in~\nfactor can scale almost linearly and achieve almost line-rate processing when scaled up to 9 runtimes. Therefore,~\nfactor has good packet processing throughput and can satisfy the stringent requirement of modern NFV system.

% reach   with a uniform We can see that the packet processing throughput scales almost linearly as the number of runtime increases, until normal case performance of running \nfactor~framework. Each flow in the generated traffic has a 10 pps (packet per second) per-flow packet rate. We vary the number of concurrently generated flows to produce varying input traffics. In this evaluation, we gradually increase the input packet rate to the \nfactor~cluster and find out the maximum packet rate that the \nfactor~cluster can support without dropping packets. In figure \ref{fig:normal-performance}, the performance of different NF modules and the service chain composed of the 3 NF modules are shown. Only one \nfactor~runtime is launched in the cluster. It is configured with different number of worker threads. In figure \ref{fig:scalability-performance}, we create different number of \nfactor~runtimes and configure each runtime with 2 worker threads. Then we test the performance using the entire service chain.

%From figure \ref{fig:normal-performance}, we can learn that the packet throughput decreases when the length of the service chain is increased. Another important factor to notice is that the \nfactor~runtime does not scale linearly as  the number of worker threads increases. The primary reason is that inside a \nfactor~runtime, there is only one packet polling thread. As the number of input packets increases, the packet polling thread will eventually become the bottleneck of the system. However, \nfactor~runtime scales almost linearly as the total number of \nfactor~runtimes increases in the cluster. When the number of runtimes is increased to 4 in the system, the maximum packet throughput is increased to 2.4M pps, which confirms to the line speed requirement of NFV system.

\subsection{Flow Migration Performance}
\label{sec:fmp}

 \begin{figure}[!t]
	\begin{subfigure}[t]{0.49\linewidth}
		\centering
		\includegraphics[width=\columnwidth]{figure/Migration.pdf}
		\caption{}\label{fig:tot-mig} \end{subfigure}\hfill
	 \begin{subfigure}[t]{0.49\linewidth}
		\centering
		\includegraphics[width=\columnwidth]{figure/Compare.pdf}
		\caption{}\label{fig:compare-opennf}
	 \end{subfigure}
\caption{ (a) The total time to migrate different numbers of flows concurrently on three runtimes. (b) The flow migration performance of NFActor. Each flow in NFActor runtime goes through the service chain consisting of the 3 customzied NF modules. OpenNF controlls PRADS asset monitors. The flow packet rate is 20pps.}
\label{fig:mig-perf}
\end{figure}

To evaluate~\nfactor's distributed flow migration performance, we configure the three runtimes on each server. Each runtime is configured with the firewall$\rightarrow$http parser service chain. The traffic generators generate flows at 20pps and the virtual switch are only configured to generate output flow packets to runtimes on the first server, therefore each runtime on the first server processes approximately the same number of flows. After the traffic stabilizes, the coordinator concurrently initiates three migrations, asking each runtime on the first runtime to migrate all of its flows to the pair runtime on the second server.

Figure~\ref{fig:tot-mig} demonstrates average flow migration completion time calculated for all three pairs of runtimes. We can see that~\nfactor~achieves excellent flow migration performance, even when migrating more than 300000 flows, with 6Mpps processing throughput. Also, the migration are concurrently executed among three pairs of runtimes, further proving the effectiveness of \nfactor's distributed flow migration. The key reason that~\nfactor is able to achieve such a good flow migration performance is because (i) flow states are directly copied to remote actor messages without the need for serialization and deserialization and (ii) remote actor messages are directly encapsulated in L2 network packet and transmitted with high-performance packet I/O.

Finally, we compare the flow migration performance of \nfactor~against OpenNF \cite{gember2015opennf}. We generate the same number of flows to both \nfactor~runtimes and NFs controlled by OpenNF and calculate the total time to migrate these flows. The evaluation result is shown in figure \ref{fig:compare-opennf}. Under both settings, the performance of~\nfactor is much better than OpenNF. Even though this is not a fair comparison, as OpenNF uses legacy NFs while~\nfactor relies on newly implemented NFs, one can still get to know the good performance achieved by~\nfactor.

%the migration completion time of \nfactor~is more than 50\% faster than OpenNF.  This performance gain primarily comes from the simplified migration protocol design with the help of actor framework. In \nfactor, a flow migration process only involves transmitting 3 request-responses. Under light workload, the flow migration can complete within several hundreds of microseconds. Under high workload, \nfactor~runtime system controls the maximum number of concurrent migrations to control the migration workload, which may increase the migration performance as indicated in figure \ref{fig:avg-time-batch-mig}. All of these factors contribute to the improved flow migration performance of \nfactor~framework.

%three runtimes and migrate flows from one runtime on the firs

%We present the evaluation result of flow migration in this section. In order to evaluate flow migration performance, we initialize the cluster with 2 runtimes running with 2 worker threads and then generate flows to one of the runtimes. Each flow is processed by the service chain consisting of all the 3 NF modules. We generate different number of flows, each flow has the same per-flow packet rate. In order to see how the evaluation performs under different per-flow packet rate, we also tune the per-flow packet rate with 10pps, 50pps and 100pps. When all the flows arrive on the migration source runtime. The migration source runtime starts migrating all the flows to the other runtime in the cluster. We calculate the total migration time and the average per-flow migration time. In order to control the workload during the migration, the runtime only allows 1000 concurrent migrations all the time. The result of this evaluation is shown in figure \ref{fig:mig-perf}.

%We can see that as the number of migrated flows increase, the migration completion time increases almost linearly. This is because the average flow migration time remains almost a constant value and the runtime controls the maximum number of concurrent migrations. Note that when the system is not overloaded at all (100 flows), the average flow migration completion time is as small as 636us.

%When the per-flow packet rate is 100pps, the maximum number of flows that we use to evaluate the system is 6000. Continuing the evaluation with 8000 and 10000 flows just overloads the runtime as shown in figure \ref{fig:normal-performance}.

 %\begin{figure}[!t]
 %\begin{subfigure}[t]{0.49\linewidth}
%		\centering
%		\includegraphics[width=\columnwidth]{figure/vary_batch_tot_migration_time.pdf}
%		\caption{The total time to migrate all the flows when changing the maximum concurrent migrations.}\label{fig:avg-time-batch-mig}
%	 \end{subfigure}\hfill
%	 \begin{subfigure}[t]{0.49\linewidth}
%	\centering
%		\includegraphics[width=\columnwidth]{figure/vary_batch_avg_migration_time.pdf}
%		\caption{The average flow migration time of a single flow when changing the maximum concurrent migrations.}\label{fig:avg-mig-batch} \end{subfigure}
%	\caption{The flow migration performance of \nfactor~when changing the maximum concurrent migrations.}
%\label{fig:mig-perf}
%\end{figure}

%Since we control the number of concurrent migrations, we also want to see what happens if we change the number of concurrent migrations. We generate 6000 flows, each with 50 pps per-flow packet rate, and change the the number of concurrent migrations. The result of this evaluation is shown in fig \ref{fig:mig-perf}. As we can see from fig \ref{fig:avg-mig-batch}, increasing the maximum concurrent migrations increase the average flow migration completion time. However, whether the total flow migration completion time increased depends on the total number of flows that wait to be migrated. From the result of fig \ref{fig:avg-time-mig}, the choice of 1000 concurrent migrations sits in the sweat spot and accelerates the overall migration process.

 %\begin{figure}[!t]
%		\centering
%		\includegraphics[width=0.6\columnwidth]{figure/opennf_nfactor_cmpFlowtime.p%df}
%		\caption{The flow migration performance of \nfactor. Each flow in \nfactor~runtime goes through the service chain consisting of the 3 customzied NF modules. OpenNF controlls PRADS asset monitors.}
%\label{fig:compare-opennf}
%\end{figure}

%Finally, we compare the flow migration performance of \nfactor~against OpenNF \cite{gember2015opennf}. We generate the same number of flows to both \nfactor~runtimes and NFs controlled by OpenNF and calculate the total time to migrate these flows. The evaluation result is shown in figure \ref{fig:compare-opennf}. Under both settings, the migration completion time of \nfactor~is more than 50\% faster than OpenNF.  This performance gain primarily comes from the simplified migration protocol design with the help of actor framework. In \nfactor, a flow migration process only involves transmitting 3 request-responses. Under light workload, the flow migration can complete within several hundreds of microseconds. Under high workload, \nfactor~runtime system controls the maximum number of concurrent migrations to control the migration workload, which may increase the migration performance as indicated in figure \ref{fig:avg-time-batch-mig}. All of these factors contribute to the improved flow migration performance of \nfactor~framework.

\subsection{Replication Performance}
\label{sec:rp}

 \begin{figure}[!t]
 \begin{subfigure}[t]{0.49\linewidth}
		\centering
		\includegraphics[width=\columnwidth]{figure/ReplicaTP.pdf}
		\caption{The packet throughput of a \nfactor~cluster when replication is enabled. }\label{fig:rep-scale}
	 \end{subfigure}\hfill
	 \begin{subfigure}[t]{0.49\linewidth}
	\centering
		\includegraphics[width=\columnwidth]{figure/Recover.pdf}
		\caption{The recovery time of three failed runtimes under different settings. The tuple on the $x$ axis represents the number of the runtime used in the evaluation and the total input packet rate. }\label{fig:rep-recovery} \end{subfigure}
	\caption{The flow migration performance of \nfactor}
\label{fig:rep-perf}
\end{figure}

To evaluate the packet processing throughput of~\nfactor's flow replication, we configure the virtual switch to generate output packets to runtimes on the first server. For each runtime on the first server, we configure a replica runtime on the second server, and let each runtime replicate its flows to the corresponding replica runtime. We calculate the cumulative packet processing throughput during flow replication. The result is shown in Figure~\ref{fig:rep-scale}. We can see that the scalability during flow replication is not as good as the result achieved when replication is disabled. The replication throughput can achieve an almost linear scalability when there are smaller than four runtimes running in a single server, but the scalability starts to degrade when more runtimes are used. The primary reason is because the bandwidth consumption on the L2 network is way higher than normal case evaluation. To ensure output-commit property, for each input packet, the runtime generates at least two output packets, containing the flow state and the output packet processed by the flow actors.

~\nfactor replicates the flow state for each processed packet using a reliable message passing module. The number of transmitted packets by the reliable message passing module is actually two times the number of the input packets. To reliably transmit flow state and the processed packet, the runtime also needs to add a header for each remote actor message, which is 52 byte long in our implementation. All these reasons contribute to a much larger bandwidth that are actually delivered over the network, which may result in potential packet drops.

In this section, we present the flow replication evaluation result. In our evaluation, the actor creates a flow snapshot for every 10 flow packets that it has processed. Then it sends the flow state snapshot to the replica storage. In this evaluation, we first generate flows to the \nfactor~cluster to test the maximum throughput of a \nfactor~cluster when enabling replication. Then we calculate the recovery time of failed \nfactor~runtime. The recovery time is the from time that the controller detects a \nfactor~runtime failure, to the time that the recovered \nfactor~finishes replaying all of its replicas and responds to the controller to rejoin the cluster. Through out this evaluation, the runtime uses the service chain consisting of the 3 NF modules to process the flow. The result of the evaluation is shown in figure \ref{fig:rep-perf}.

In figure \ref{fig:rep-scale}, we can see that there is an obvious overhead to enable replication on \nfactor~runtimes. The overall throughput when replication is enabled drops around 60\%. This is due to the large amount of replication messages that are exchanged during the replication process. Internally, the replication messages are sent over Linux kernel networking stack, which involves data copy and context switching, thus increasing the performance overhead of using replication. However, the overall throughput when replication is enabled could scale to 850K pps when 4 runtimes are used, which is enough to use in some restricted settings.

Finally, figure \ref{fig:rep-recovery} shows the recovery time of \nfactor~runtime when replication is enabled. We found that the recovery time remains a consistent value of 3.3s, no matter how many runtimes are used or how large the input traffic is. The reason of this consistent recovery time is that the \nfactor~runtime maintains one replica on every other \nfactor~runtimes in the cluster. During recovery, several recovery threads are launched to fetch only one replica from another runtime. Then each recovery thread independently recovers actors by replaying its own replica. In this way, the recovery process is fully distributed and scales well as the number of replica increases. Note is that the average time it takes for a recovered runtime to fetch all the replicas and recover all of its actors is only 1.2s. So actually around 2.1s is spent in container creation and connection establishment.

\begin{figure}[!h]
	\centering
	\includegraphics[width=\columnwidth]{figure/Scale.pdf}
	\caption{The packet processing throughput using different number of runtimes. The runtimes are configured with different NFs and different service chains.
  \textbf{FM}: Flow Monitor. \textbf{HP}: HTTP Parser. \textbf{FW}: Firewall.}
\label{fig:normal-case-eval}
\end{figure}

\begin{figure*}[!ht]
\begin{subfigure}[t]{0.33\linewidth}
   \centering
   \includegraphics[width=\columnwidth]{figure/Dynamic.pdf}
   \caption{The packet throughput of a \nfactor~cluster when replication is enabled. }\label{fig:rep-scale}
  \end{subfigure}\hfill
  \begin{subfigure}[t]{0.33\linewidth}
 \centering
   \includegraphics[width=\columnwidth]{figure/Dedup.pdf}
   \caption{The recovery time of three failed runtimes under different settings. The tuple on the $x$ axis represents the number of the runtime used in the evaluation and the total input packet rate. }\label{fig:rep-recovery} \end{subfigure}\hfill
   \begin{subfigure}[t]{0.33\linewidth}
  \centering
    \includegraphics[width=\columnwidth]{figure/MPTCP.pdf}
    \caption{The recovery time of three failed runtimes under different settings. The tuple on the $x$ axis represents the number of the runtime used in the evaluation and the total input packet rate. }\label{fig:rep-recovery} \end{subfigure}
 \caption{The flow migration performance of \nfactor}
\label{fig:rep-perf}
\end{figure*}

%\section{Discussion}
\label{sec:discussion}

Even though~\nfactor provides transparent resilience for stateful NFs,~\nfactor focuses on handling per-flow state. Currently,~\nfactor could not correctly handle shared states, \ie, the states shared by a bunch of flows. Even though the NF API in~\nfactor achieves a clean separation between per-flow state and NF processing logic, it can not correctly separate shared state. Therefore, migrating and replicating flows that share states with other flows may cause un-predicted errors in~\nfactor. A potential solution to this limitation is to enforce the programmer to write a handler that explicitly deals with the inconsistency during resilience operation. We leave this to our future work.

Another limitation of~\nfactor is that~\nfactor may incorrectly handle flows with packet encapsulation. \nfactor uses the flow-5-tuple to differentiate flows. However, different flows may share the same flow-5-tuple if their flow packets are encapsulated. This is a common for flows that are sent over the same VxLAN tunnel. In that case, those flows are handled by the same flow actor, resulting in incorrect flow processing. If~\nfactor knows what kind of encapsulation the input packet uses,~\nfactor could add a decapsulation function in the virtual switch to correctly extract different flows. This is also left in our future work.

To achieve transparent resilience,~\nfactor~requires NF to be rewritten a new set of API to achieve clean separation between flow state and NF core logic, making legacy NFs difficult to run on~\nfactor. However, with the development of NFV system, there is a practical need for people to create new NFs. NFs that process flows based on flow state could achieve transparent resilient if they are implemented using~\nfactor.

The detailed flow replication process is illustrated in Fig.~\ref{fig:flow-rep}. When a flow actor is created, it acquires its replication target runtime by sending a local actor message to liason actor.  %xxx\chuan{tell which entity the flow actor query, and through which messaging approach, remote message passing or RPC?}.
The liason actor sends back a local actor message containing the ID
%\chuan{what of the replication target runtime to be sent to the requesting flow actor?}
of the replica runtime in a local actor message, selected in the round-robin fashion among all available runtimes in the same cluster which run on a different physical server from where the server hosting the runtime. %\chuan{describe what if there is no other runtime in the cluster?}.
The coordinator monitors the number of available replica runtime in the cluster and launches new replica runtimes if there are no available replica runtime.  After the flow actor acquires a valid replica runtime, whenever it finishes processing an input flow packet %\chuan{what packet is this? the first flow packet?}
, it sends a replication actor message, containing the current flow states and the input packet %\chuan{do you only need to send the 5 tuples?}
, directly to the liaison actor on the replication target runtime. The liaison actor uses the flow-5-tuple contained in the input packet of the replication actor message to check whether there exists a replica flow actor on the replication target runtime. If not, it creates a new replica flow actor using the same flow-5-tuple contained in the replication message and forwards all subsequent replication messages that share the same flow-5-tuple to that replica flow actor, which saves the flow states contained in the replication message and sends the input packet out from the output port of the replica runtime.
%The liaison actor on the replication target runtime creates a replica flow actor using the same flow-5-tuple contained in the first packet as the original flow actor, in preparation to handle all the replication messages. The replica flow actor saves the flow state and sends first packet out from the output port of the replica runtime. Subsequent flow packet received by the original flow actors are %\chuan{is that the packet is sent out from the replica instead of original runtime even if the later has not failed? Why?}.
Similar with \cite{sherry2015rollback}, the receiver on the side of the output port of the replica runtime can only observe an output packet when the flow state has been replicated, which guarantees the same output-commit property as in \cite{sherry2015rollback}. %\chuan{I think only after an original runtime fails, packets will be sent out from the replica}.

\section{Related Work}
\label{sec:relatedwork}

%Discuss related work here.

%\chuan{remember to discuss the EuroSys’16 work ``Optimizing Distributed Actor Systems for Dynamic Interactive Services''}


\textbf{Network Function Virtualization (NFV).} NFV is a new trend that advocates moving from running hardware middleboxes to running software network function instances in virtualized environment. The literature has developed a broad range of NFV applications, from scaling and controlling the NFV systems \cite{gember2012stratos, palkar2015e2}, to improving the performance of NFV software \cite{hwang2015netvm, Han:EECS-2015-155, martins2014clickos, 199352}, to migrating flows among different NF instances \cite{rajagopalan2013split, khalid2016paving, gember2015opennf}, and to replicating NF instances \cite{rajagopalan2013pico, sherry2015rollback}. However, none of the above mentioned systems provide a uniform runtime platform to execute network functions. Most of the NF instances are still created as a standalone software running inside virtual machine or containers. Even though modular design introduced by ClickOS \cite{kohler2000click} simplifies the way of how NF functions are constructed, however, nowadays there are new demands for NFV system, which require advanced control functionality to be integrated even into the NF softwares. 

Among the advanced control functionality, flow migration and fault tolerance are definitely the two of the most important features. Existing work such as OpenNF \cite{gember2015opennf} and Split/Merge \cite{rajagopalan2013split} requires direct modification to the core processing logic of NF softwares, which is tedious and hard to do. On the other hand, existing work rely on SDN to carry out migration protocol, thereby increasing the complexity of the migration protocol. Finally, the migration process is fully controlled by a  centralized SDN controller, which may not be scalable if there are many NF instances that need flow migration service. The proposed NFActor framework overcomes most of the above mentioned obstacles by providing a uniform runtime system constructed with actor framework. The actors could be migrated by themselves without the coordination from a centralized controller. The framework provides a  fast virtual switch to substitute the functionality of a dedicated SDN switch. With the help of the actor framework and the customized virtual switch, the migration protocol only needs to transmit 3 request-responses. Finally, the NFActor achieves transparent migration without the need for manual modification of the NF software. This greatly simplifies the the required procedures for using migration service.

Another important control functionality lies on replication. The replication process usually involves check-pointing the entire process image and making a back-up for the created process image \cite{sherry2015rollback}, which may halt the execution of the NF software, leading to packet losses. NFActor framework is able to check-point of the state of the flow, which is relatively lightweight to do and does not incur a high latency overhead. Similar with migration process, NF modules written using NFActor framework could be transparently replicated. Existing work like \cite{sherry2015rollback} rely on automated tools to extract important state variables for replicating.

\textbf{Actor Programming Model.} The actor programming model has been widely used to construct resilient distributed software \cite{erlang, akka, Orleans, caf}. The actors are asynchronous entities that can receive and send messages as if they are running in a dedicated process. The actors usually run on a powerful runtime system \cite{erlang, akka, caf}, enabling them to achieve network transparency. It greatly simplifies programming with actor model. Even though actor programming model is widely used in both the industry and academic worlds, we have not found any related work that leverage actor programming model to construct NFV system, even though there is a natural connection among actor message processing and NF flow processing. Reliazing this problem, we are the first one to introduce actor programming model into NFV system and shows that using actor programming model can really bring benefits for designing NFV applications.

\textbf{Lightweight Execution Context. } There has been a study on constructing lightweight execution context \cite{litton2016light} in kernel. In this work, the authors construct a light weight execution context by creating multiple memory mapping table in the same process. Switching among different memory tables could be viewed as switching among different lightweight execution contexts. NFActor provides a similar execution context, not for kernel processes, but for network functions. Each actor inside NFActor framework actually provides a lightweight execution context for processing a packet along a service chain. Being a lightweight context, the actors do not introduce too much overhead as we can see from the experiment session. On the other hand, packet processing is fully monitored by the execution context, thereby providing a transparent way to migrate and replicate flow states.









\section{Conclusion}
\label{sec:conclusion}

In this work, we present a new framework for building resilient NFV system, called NFActor framework. Unlike existing NFV system, where NF instances run as a program inside a virtual machine or a container, NFActor framework provides a set of API to implement NF modules which executes on the runtime system of NFActor framework. Inside the NFActor framework, packet processing of a flow is dedicated to an actor. The actor provides an execution context for processing packets along the service chain, reacting to flow migration and replication messages. NF modules written using the API provided by NFActor framework achieves flow migration and state replication functionalities in a transparent fashion. The implementer of the NF module therefore only needs to concentrate on designing the core logic. Evaluation result shows that even though the NFActor framework incurs some overhead when processing packets, the scalability of NFActor runtime is good enough to support line-speed requirement. NFActor framework outperforms existing works by more than 50\% in flow migration completion time. Finally, the flow state replication of NFActor is scalable and achieves consistent recovery time.

\bibliographystyle{abbrv}
\bibliography{bibliography}

\end{document}
