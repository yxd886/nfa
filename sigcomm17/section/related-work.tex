\section{Related Work}
\label{sec:relatedwork}

%Discuss related work here.

%\chuan{remember to discuss the EuroSys’16 work ``Optimizing Distributed Actor Systems for Dynamic Interactive Services''}


\textbf{Network Function Virtualization (NFV).} NFV is a new trend that advocates moving from running hardware middleboxes to running software network function instances in virtualized environment. The literature has developed a broad range of NFV applications, from scaling and controlling the NFV systems \cite{gember2012stratos, palkar2015e2}, to improving the performance of NFV software \cite{hwang2015netvm, Han:EECS-2015-155, martins2014clickos, 199352}, to migrating flows among different NF instances \cite{rajagopalan2013split, khalid2016paving, gember2015opennf}, and to replicating NF instances \cite{rajagopalan2013pico, sherry2015rollback}. However, none of the above mentioned systems provide a uniform runtime platform to execute network functions. Most of the NF instances are still created as a standalone software running inside virtual machine or containers. Even though modular design introduced by ClickOS \cite{kohler2000click} simplifies the way of how NF functions are constructed, however, nowadays there are new demands for NFV system, which require advanced control functionality to be integrated even into the NF softwares. 

Among the advanced control functionality, flow migration and fault tolerance are definitely the two of the most important features. Existing work such as OpenNF \cite{gember2015opennf} and Split/Merge \cite{rajagopalan2013split} requires direct modification to the core processing logic of NF softwares, which is tedious and hard to do. On the other hand, existing work rely on SDN to carry out migration protocol, thereby increasing the complexity of the migration protocol. Finally, the migration process is fully controlled by a  centralized SDN controller, which may not be scalable if there are many NF instances that need flow migration service. The proposed NFActor framework overcomes most of the above mentioned obstacles by providing a uniform runtime system constructed with actor framework. The actors could be migrated by themselves without the coordination from a centralized controller. The framework provides a  fast virtual switch to substitute the functionality of a dedicated SDN switch. With the help of the actor framework and the customized virtual switch, the migration protocol only needs to transmit 3 request-responses. Finally, the NFActor achieves transparent migration without the need for manual modification of the NF software. This greatly simplifies the the required procedures for using migration service.

Another important control functionality lies on replication. The replication process usually involves check-pointing the entire process image and making a back-up for the created process image \cite{sherry2015rollback}, which may halt the execution of the NF software, leading to packet losses. NFActor framework is able to check-point of the state of the flow, which is relatively lightweight to do and does not incur a high latency overhead. Similar with migration process, NF modules written using NFActor framework could be transparently replicated. Existing work like \cite{sherry2015rollback} rely on automated tools to extract important state variables for replicating.

\textbf{Actor Programming Model.} The actor programming model has been widely used to construct resilient distributed software \cite{erlang, akka, Orleans, caf}. The actors are asynchronous entities that can receive and send messages as if they are running in a dedicated process. The actors usually run on a powerful runtime system \cite{erlang, akka, caf}, enabling them to achieve network transparency. It greatly simplifies programming with actor model. Even though actor programming model is widely used in both the industry and academic worlds, we have not found any related work that leverage actor programming model to construct NFV system, even though there is a natural connection among actor message processing and NF flow processing. Reliazing this problem, we are the first one to introduce actor programming model into NFV system and shows that using actor programming model can really bring benefits for designing NFV applications.

\textbf{Lightweight Execution Context. } There has been a study on constructing lightweight execution context \cite{litton2016light} in kernel. In this work, the authors construct a light weight execution context by creating multiple memory mapping table in the same process. Switching among different memory tables could be viewed as switching among different lightweight execution contexts. NFActor provides a similar execution context, not for kernel processes, but for network functions. Each actor inside NFActor framework actually provides a lightweight execution context for processing a packet along a service chain. Being a lightweight context, the actors do not introduce too much overhead as we can see from the experiment session. On the other hand, packet processing is fully monitored by the execution context, thereby providing a transparent way to migrate and replicate flow states.








