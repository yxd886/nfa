\section{Introduction}


Network function virtualization (NFV) advocates moving
{\em network functions} (NFs) out of dedicated hardware middleboxes and running them as
virtualized applications on commodity servers \cite{nfv-white-paper}. With NFV, network
operators %no longer need to maintain complicated and costly hardware middleboxes. Instead, they may
can launch virtualized devices (virtual machines or containers) to run NFs on the fly %which drastically reduces the cost and complexity of
 and provision network services. A network service typically consists of a sequence of NFs for flow processing, described by a {\em service chain}, \eg, ``firewall $\rightarrow$ intrusion detection system (IDS) $\rightarrow$ loader balancer''.

 %However, simply running NF software in virtualized environment is not enough to satisfy the stringent requirement of NFV. What network operators need is a full-fledged NFV system, that is capable of handling different kinds of NFV management tasks.
A number of NFV management systems have been designed in recent years, \eg, E2 \cite{palkar2015e2}, OpenBox \cite{OpenBox}, CoMb
\cite{sekar2012design}, xOMB \cite{anderson2012xomb}, Stratos
\cite{gember2012stratos} and OpenNetVM \cite{zhang2016opennetvm}. %and ClickOS \cite{martins2014clickos}. %\chuan{comment out from the list the none `management' systems}.
They implement a
broad range of management functionalities, including %line-speed packet processing \chuan{it is not management functionality},
 dynamic NF placement, elastic NF scaling,
load balancing, etc. %, which facilitate network operators in operating NF service chains in virtualized environments.
 However, failure
tolerance \cite{rajagopalan2013pico, sherry2015rollback} capabilities of the NF systems are always lacking, together with efficient support for flow migration \cite{gember2015opennf, rajagopalan2013split, khalid2016paving}.%both of which are of pivotal importance in practical NFV systems for resilience and scalability.
%\cui{add dpdk}

{\em Failure tolerance is crucial for stateful NFs.}  Many NFs maintain important per-flow states \cite{EnablingNF}: IDSs such as Bro \cite{bro} %parse different network/application protocols, and
 store and update protocol-related states for each flow to alert potential attacks; firewalls \cite{firewall} parse TCP SYN/ACK/FIN packets and maintain TCP connection related states for
each flow; load balancers \cite{lvs} may retain mapping between a flow identifier and
the server address, for modifying destination address of packets in the flow.
It is critical to ensure correct recovery of flow states in case of NF failures, such that the connections handled by the failed NFs do not have to be reset -- a simple approach strongly rejected by middlebox vendors \cite{sherry2015rollback}.
%strongly rejected the idea of simply resetting all active connections after failure as it disrupts users \cite{sherry2015rollback}.

{\em Efficient flow migration is important for long-lived flows for dynamic system scaling.} Existing NF systems \cite{palkar2015e2, gember2012stratos} mostly assume dispatching new flows to newly created NF instances when existing instances are overloaded, or waiting for remaining flows to complete before shutting down a mostly idle instance, %\chuan{add citations to these systems},
which are only efficient in cases of short-lived flows. Long flows are common in the Internet: %Web applications %such as websites and on-line games
% usually multiplex application-level requests and responses in one TCP connection to improve performance. For example,
a web browser uses one TCP connection to exchange many requests and responses with a
web server \cite{http-keep-alive}; video-streaming
\cite{ffmpeg} and file-downloading \cite{ftp} systems maintain long-lived TCP
connections for fetching large volumes of data from CDN servers. %These applications contribute to many long-lived flows.
 When NF instances handling long flows are overloaded or under-loaded, migrating flows to other available NF instances enables timely hotspot resolution and system scaling \cite{gember2015opennf}. %; when some NF instances are handling a few dangling long flows each, it is also more resource/cost effective to migrate the flows to one NF instance while shutting the others down.
%Without flow migration, the overload of NF instances could not be mitigated in a timely manner because of long flows \cite{gember2015opennf}.


%Given the importance of failure resilience and flow migration in an NFV system,
Why are failure tolerance and efficient flow migration missing in the existing NFV systems? The reason is simple: enabling them % flow migration and fault tolerance
 has been a challenging task on the existing NF software architectures. To provide resilience and enable flow migration, important NF states must be correctly extracted from NF software for transmitting to a backup NF instance. %, needed for both flow migration and replication (for resilience).
 However, a separation between NF states and core processing logic is not enforced in the state-of-the-art implementation of NF software. Especially, important NF states may be scattered across the code base of the software, making
extracting and serializing these states a daunting task. Patch code needs to be
manually added to the source code of different NFs for this purpose %to extract and serialize NF states
 \cite{gember2015opennf}\cite{rajagopalan2013split}, which usually requires a huge amount of manual work to add up to
thousands of lines of code for one NF. For example, Gember-Jacobson {\em et al.}~\cite{gember2015opennf} report
that 3.3K lines of code is needed for patching Bro \cite{bro} and 7.8K for Squid caching
proxy \cite{squid}.  Realizing this difficulty, Khalid {\em et al.}~\cite{khalid2016paving} use
static program analysis technique to automate this process. However, applying
static program analysis itself is a challenging task and the inaccuracy of
static program analysis may prevent some important NF states from being
correctly retrieved.

Even if NF states can be correctly acquired and NF replicas created, flows need to be redirected to the new NF instances in cases of load balancing and failure recovery. The existing systems typically handle this using a centralized SDN controller \cite{gember2015opennf, rajagopalan2013split}. %\chuan{add more references to these systems}.
The controller initiates and coordinates the entire migration process of each flow, which involves multiple messages passes to ensure losslessness, leading to compromised scalability and additional delay. %for lossless flow migration, the controller has to perform complicated migration protocols that involve multiple passes of messages among the SDN controller, switches, migration source and migration target \cite{gember2015opennf}, which adds delay to flow processing and limits packet processing throughput of the system.

%Even if NF states can be correctly acquired, transmitting
%the states among different NFs %and the NFV system controller
% requires an
%effective message passing service. The existing NF software (\eg,
%Click\cite{kohler2000click}) does not usually provide the support for a messaging channel, and programmers need to manually add this communication
%channel into the NF software. Finally, the additional codes that are patched to
%implement resilience inevitably entangle with the core processing logic of NF
%software. It may lead to serious software bugs if not handled properly.


This paper presents a software framework for building resilient NFV systems, \nfactor, exploiting the distributed actor model \cite{actor-wiki, akka, newell2016optimizing} for efficient flow migration, replication and system scaling. %Our main observation is that
Especially, the actor model enables %provides the unique benefits for
 lightweight extraction and decentralized migration of network flow states, based on which we enable highly efficient flow migration and replication.
% \nfactor~tracks each flow's state with our high-performance flow actor, whose design transparently separates flow state from NF processing logic. \nfactor~provides service chain processing of flows using flow actors on carefully designed uniform runtime environment, and enables fast flow migration and replication without relying much on centralized control.
%\nfactor~uses actors to process flow packets at a high throughput rate. The flow states are kept by the actors, which can be migrated without a centraliezd coordinator.
%\nfactor~includes the following modules: (i) \ac{a light-weight controller for basic cluster management} (ii) \ac{runtimes that conduct service chain processing using actors} (iii) \ac{virtual switches for balancing the workload on runtimes.} \chuan{briefly introduce key modules in nfactor system}
Transparent resilience, agile scalability and high efficiency in network flow processing are achieved in \nfactor~based on the following design highlights.

$\triangleright$ {\em Clean separation between NF processing logic and resilience enabling operations.} %Unlike existing work \cite{gember2015opennf, sherry2015rollback} that patch functionalities for failure resilience into NF software,
 We design a highly efficient flow actor architecture, which provides a clean separation between important NF states and core NF processing logic in a service chain using a set of easy-to-use APIs. Extracting and transmitting flow states hence become an easy task. Based on this, \nfactor~can carry out flow migration and replication operations, those needed to enable failure resilience, transparently to concrete NF processing, which we refer to as {\em transparent resilience}. %In \nfactor, programmers implementing the NFs only need to focus on the core NF logic, and the framework provides the resilience support. % On the other hand, flow actors can be transparently migrated and replicated as long as they load NF modules that are written with \nfactor~API. We refer to this as transparent resilience in~\nfactor.


$\triangleright$ {\em Per-flow micro management.} Fundamentally different from the existing systems, \nfactor~creates a micro execution context for each flow by providing a dedicated flow actor for processing the flow through its entire service chain, \ie, %This can be viewed as
a micro service chain service dedicated to the flow.
 %Actors in \nfactor~are configured with a rich set of message handlers and run on uniform runtime systems, enabling direct communication between remote actors running on different runtime systems.
%\chuan{idea of the following sentence has been covered by the paragraph above; instead you should describe how this per-flow management can enable better scalability and high speed packet processing}
%The micro execution context is constructed using actor framework, which has been proven to be a light-weight and scalable abstraction for building high-performance systems \cite{newell2016optimizing}.
 Multiple flow actors run within one runtime system (\eg, one container), with lightweight, efficient actor scheduling to achieve high packet processing throughput. \nfactor~consists of multiple uniform runtime systems, that we have carefully designed to provide high system scalability and easy flow replication. %of \nfactor is also improved as actors can be scheduled to run on uniform runtime systems.
%Inside this execution context, the flow actor can actively exchange messages with other actors and transmit flow states without disturbing the normal NF processing.


%NFs in a service chain are deployed inside the execution context of an actor, instead of being chained through different virtualized devices (\eg, virtual machines or containers). In this way, a unique actor is responsible for processing a network flow through a dedicated service chain. This unique actor fully monitors the flow processing. It can interrupt the flow processing for flow migration or fault tolerance without the need to contact the service chain.


$\triangleright$ {\em Largely decentralized flow migration and replication.} Based on the actor framework, flow migration and replication processes in \nfactor~are automated through decentralized message passing. A central coordinator is involved only during initialization stage of flow migration and replication. % achieved in a fully distributed fashion without continuous monitoring of a centralized controller, which distinguishes \nfactor~from the existing NFV systems \cite{gember2015opennf}. The controller in \nfactor~is only used for controlling dynamic scaling and initiating flow migration and replication,
 In addition, runtimes in \nfactor use the high-speed packet I/O library DPDK \cite{dpdk} for retrieving/transmitting flow packets and control messages for flow migration and replication, achieving high efficiency.
%The actor programming model only imposes a small overhead during service chain processing and flow management, improving the performance of NFActor.


We implement \nfactor~and open source the project \cite{projectcode}.
%\chuan{improve the result discussion}
Our testbed experiments show the good scalability of \nfactor to approach line-rate packet processing, zero packet loss during concurrent migration of more than 100K flows, and recovery of tens of thousands of flows within tens of milliseconds in case of runtime failures. %performance of the runtime system is good and can be scaled up to support line-rate throughput. 
% The flow migration supports concurrent migration of more than 100K flows, while the distributed migration guarantees zero packet loss. The flow replication is scalable, achieves desirable throughput and enables a failed runtime to be recovered within tens of milliseconds. The dynamic scaling of~\nfactor uses flow migration to quickly resolve hotspot and avoid throughput loss. The new applications enabled by the distributed flow migration achieve significant good result when compared with traditional method.
Going beyond resilience, we also show that a couple of interesting applications can be efficiently enabled on \nfactor, live NF update and correct MPTCP subflow processing. These applications require individual NFs to initiate flow migration, %explicitly notify the flow actor to migrate,
which is hard to achieve (without significant overhead) in the existing systems. % where flow migration is initiated and fully monitored by a centralized controller.
Our decentralized and fast flow migration enable these applications with ease. %to achieve live NF update with almost no interruption to high-speed packet processing of the NF, best flow deduplication to conserve bandwidth, and correct MPTCP subflow processing, with ease. %NFActor framework also provides live updates to NFs that process packets at the rate of millions packets per second with almost zero down time, due to the blazingly fast flow migration speed.

%The rest of the paper is organized as follows. \chuan{to complete} %We present background about NFV and the actor model in Sec.~\ref{sec:background} and overview our \nfactor~system in Sec.~\ref{sec:overview}. We discuss in detail the fault tolerance and flow migration design in Sec.~\ref{sec:fm} and Sec.~\ref{sec:ft}. We show the implementation and evaluation results in Sec.~\ref{sec:implementation} and Sec.~\ref{sec:experiments}, followed by related work in Sec.~\ref{sec:relatedwork}. Sec.~\ref{sec:conclusion} concludes the paper.







%Unfortunately, few existing NFV systems could achieve the goal to resiliently maintain network functions. In most of the existing NFV management systems (i.e. E2 \cite{palkar2015e2}, OpenBox \cite{bremler2015openbox}), network functions have been treated as a black box, which consume packets from ingress ports and generate output packets from egress ports. Therefore these systems only provide a per-NF based management scheme. Even though per-NF based management has been proved to be effective in dealing with dynamic scaling and NF planning, it might compromise the reliability and resilience under certain circumstances. A typical example is during the update to important NF configuration files (i.e. Firewall rule) or to the NF softwares, the NF instances usually need to be shutdown. Due to the limitation of per-NF based management, there is no effective method to prevent established network flows from being forced to shutdown due to this process. The only way towards solving this problem and creating a fully resilient NFV system is to provide efficient per-flow based management, on top of which to achieve flow migration and replication for true resilience.
%In this paper, we propose a new NFV building framework, called NFActor.

%However, several open problems have existed with per-flow based management scheme. It is hard to migrate flows lively without direct support from NF runtime system. Therefore, in per-NF based NFV systems, migrating flows by directly changing the route of the flow may cause serious packet drop and may lead to inconsistent flow states. Several existing works including OpenNF \cite{gember2015opennf} and Split/Merge \cite{rajagopalan2013split} made initial contributions to provide a runtime that supports live flow migration. However, the scalability of their approach is limited by using a centralized controller to coordinate the entire flow migration process. The centralized controller may also be a single point of failure, making their systems vulnerable to software bugs. Also, in order to use their runtime systems, people need to add non-trivial patch code to existing NF softwares, compromising the usability of their systems.

%However, with the developement of NFV, researchers found out that managing at middlebox level could not satisfy the requirement of some applications. Some applications require direct management of a single network flow. A straightforwad example is flow migration. When migrating a flow, the NFV management system must transfer the state information associated with the flow from one middlebox to another, and redirecting the flow to the new middlebox in the mean time. Another example is fault tolerance of an individual flow. The NFV management system has to replicate flow's state on a replica and recovers flow's state on a new middlebox in case of the failure of the old middlebox.

%Realizing these limitations, we propose a new NFV management system in this paper, called NFActor. NFActor framework is designed to provide transparent, scalable and efficient flow management. NFActor framework achieves this goal by creating a micro execution context, running on top of a uniform runtime system, for each flow using actor programming model. This execution context is augmented with several message handlers to achieve basic service chain processing and flow management tasks. To transparently integrate NF softwares with the execution context, we provide a new programming interface for creating new NF modules for NFActor framework, which enforces separation between the core NF processing logic and the NF state of each flow.

%Using NFActor framework, programmers could concentrate on the internal logic design of the NF. NFs written using the NFActor programming interface could be transparently integrated with the flow management tasks provided by the actor execution context. Due to the message passing and decentralized nature of actor programming model, the flow management tasks are fully automated by actor scheduling and message passing. There is no need for the continuous monitoring of a centralized controller. Therefore the controller in NFActor framework is extremely light-weighted and failure resilient. The actor programming model only imposes a small overhead during service chain processing and flow management, improving the performance of NFActor.

%Besides strong resilience through per-flow management, NFActor framework also enables several interesting new applications that existing NFV systems are hard to provide. These applications utilize the feature of decentralized flow migration to reduce the output bandwidth consumption during deduplication and ensures correct MPTCP subflow processing. NFActor framework also provides live updates to NFs that process packets at the rate of millions packets per second with almost zero down time, due to the blazingly fast flow migration speed.

%We implement NFActor framework on top of DPDK and evaluate its performance. The result shows that the performance of the runtime system is desirable. The runtimes have almost linear scalbility. The flow migration is blazingly fast. The flow replication is scalable, achieves desirable throughput and recover fast. The dynamic scaling of NFActor framework is good with flow migration. The result of the applications are good and positive.
