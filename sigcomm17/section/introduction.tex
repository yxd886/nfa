\section{Introduction}

The recent paradigm of Network Function Virtualization (NFV) eases the network maintenance tasks by moving Network Functions (NFs) out of dedicated hardware middleboxes and running them as virtualized applications on commodity servers \cite{nfv-white-paper}. With NFV, network operators no longer need to maintain complicated and costly hardware middleboxes. Instead, they can launch virtualized devices (virtual machines or containers) to run NFs on the fly, which drastically reduces the cost and complexity of deploying network services, usually consisting of a sequence of NFs such as ``firewall$\rightarrow$IDS$\rightarrow$proxy''.


Unfortunately, few existing NFV systems could achieve the goal to resiliently maintain network functions. In most of the existing NFV management systems (i.e. E2 \cite{palkar2015e2}, OpenBox \cite{bremler2015openbox}), network functions have been treated as a black box, which consume packets from ingress ports and generate output packets from egress ports. Therefore these systems only provide a per-NF based management scheme. Even though per-NF based management has been proved to be effective in dealing with dynamic scaling and NF planning, it might compromise the reliability and resilience under certain circumstances. A typical example is during the update to important NF configuration files (i.e. Firewall rule) or to the NF softwares, the NF instances usually need to be shutdown. Due to the limitation of per-NF based management, there is no effective method to prevent established network flows from being forced to shutdown due to this process. The only way towards solving this problem and creating a fully resilient NFV system is to provide efficient per-flow based management, on top of which to achieve flow migration and replication for true resilience.
%In this paper, we propose a new NFV building framework, called NFActor.

However, several open problems have existed with per-flow based management scheme. It is hard to migrate flows lively without direct support from NF runtime system. Therefore, in per-NF based NFV systems, migrating flows by directly changing the route of the flow may cause serious packet drop and may lead to inconsistent flow states. Several existing works including OpenNF \cite{gember2015opennf} and Split/Merge \cite{rajagopalan2013split} made initial contributions to provide a runtime that supports live flow migration. However, the scalability of their approach is limited by using a centralized controller to coordinate the entire flow migration process. The centralized controller may also be a single point of failure, making their systems vulnerable to software bugs. Also, in order to use their runtime systems, people need to add non-trivial patch code to existing NF softwares, compromising the usability of their systems.

%However, with the developement of NFV, researchers found out that managing at middlebox level could not satisfy the requirement of some applications. Some applications require direct management of a single network flow. A straightforwad example is flow migration. When migrating a flow, the NFV management system must transfer the state information associated with the flow from one middlebox to another, and redirecting the flow to the new middlebox in the mean time. Another example is fault tolerance of an individual flow. The NFV management system has to replicate flow's state on a replica and recovers flow's state on a new middlebox in case of the failure of the old middlebox.

Realizing these limitations, we propose a new NFV management system in this paper, called NFActor. NFActor framework is designed to provide transparent, scalable and efficient flow management. NFActor framework achieves this goal by creating a micro execution context, running on top of a uniform runtime system, for each flow using actor programming model. This execution context is augmented with several message handlers to achieve basic service chain processing and flow management tasks. To transparently integrate NF softwares with the execution context, we provide a new programming interface for creating new NF modules for NFActor framework, which enforces separation between the core NF processing logic and the NF state of each flow.

Using NFActor framework, programmers could concentrate on the internal logic design of the NF. NFs written using the NFActor programming interface could be transparently integrated with the flow management tasks provided by the actor execution context. Due to the message passing and decentralized nature of actor programming model, the flow management tasks are fully automated by actor scheduling and message passing. There is no need for the continuous monitoring of a centralized controller. Therefore the controller in NFActor framework is extremely light-weighted and failure resilient. The actor programming model only imposes a small overhead during service chain processing and flow management, improving the performance of NFActor.

Besides strong resilience through per-flow management, NFActor framework also enables several interesting new applications that existing NFV systems are hard to provide. These applications utilize the feature of decentralized flow migration to reduce the output bandwidth consumption during deduplication and ensures correct MPTCP subflow processing. NFActor framework also provides live updates to NFs that process packets at the rate of millions packets per second with almost zero down time, due to the blazingly fast flow migration speed.

We implement NFActor framework on top of DPDK and evaluate its performance. The result shows that the performance of the runtime system is desirable. The runtimes have almost linear scalbility. The flow migration is blazingly fast. The flow replication is scalable, achieves desirable throughput and recover fast. The dynamic scaling of NFActor framework is good with flow migration. The result of the applications are good and positive.

%The key contributions of this papers are 3 folds:

%\begin{itemize}

%\item A new framework for building resilient NFV systems that are resilient in nature.

%\item The design and efficient implementation of this framework using DPDK.

%\item Several new applications that are inspired with our new framework.

%\end{itemize}
