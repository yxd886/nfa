\section{Background}
\label{sec:background}

\subsection{Network Function Virtualization}

A NFV system \cite{nfv-white-paper} typically consists of a controller and
many VNF instances. Each VNF instance is a virtualized device running NF software. VNF instances are connected into service chains, implementing certain network services, \eg, access service. Packets of a network flow go through the NF instances in a service chain in order before reaching the destination.

A VNF instance constantly polls a network interface card (NIC) for packets. Using traditional kernel network stack incurs high context switching overhead \cite{martins2014clickos} and greatly compromise the packet processing throughput. To speed things up, hypervisors usually map the memory holding packet buffers directly into the address space of the VNF instances with the help of Intel DPDK\cite{dpdk} or netmap \cite{netmap}. VNF instances then directly fetch packets from the mapped memory area, avoiding expensive context switches. Recent NFV systems \cite{palkar2015e2, Han:EECS-2015-155, sherry2015rollback, martins2014clickos, hwang2015netvm} are all built using similar techniques.

%尽管使用DPDK或netmap来实现高速的包处理已经成为了一种必然的趋势,现有的流管理系统仍然在使用kernel networking stack来实现communication channel。NFActor则完全抛弃了kernel networking stack。NFActor利用DPDK构建了自己的可靠传输系统。这套可靠传输系统可实现6M/s的消息传输吞吐量。更重要的是,使用这套可靠传输系统不会引发任何的context switching,这进一步提高了整个系统的速度。 我们利用一个中央化的调度器将可靠传输模块与其他模块无缝的衔接起来

Even though using DPDK and netmap to improve the performance of packet processing has become a new trend. Existing flow management systems are still using kernel networking stack to implement the communication channel. On contrary, NFActor completely abandons the kernel networking stack, by constructing a reliable transmission module using DPDK. Using this reliable transmission module does not incur any context switches, thereby boosting the message throughput to 6 million messages per second in our evalution.

\subsection{Actor Model}

The actor programming model has been used as the basic building block for constructing massive, distributed systems\cite{actor-wiki, akka, newell2016optimizing}. Each actor is an independent execution unit, which can be viewed as a logical thread. In the simplest form, an actor contains an internal actor state (\eg, statistic counter, status of peer actors), a mailbox for accepting incoming messages and several message handler functions. An actor can process incoming messages using its message handlers, send messages to other actors through the built-in message passing channel, and create new actors.

There are several popular actor frameworks, \ie, Scala Akka \cite{akka}, Erlang \cite{erlang}, Orleans \cite{Orleans} and C++ Actor Framework \cite{caf}. These actor frameworks have been used to build a broad range of distributed programs, including on-line games and e-commerce. For example, Blizzard (a famous PC game producer) and Groupon/Amazon/eBay (famous e-commerce websites) all use Akka in their production environment \cite{akka}.

%actor model可以被很自然的拿来构建flow的执行环境。在一个VNF instance里,我们为每一个flow都创建一个actor. 并将flow包的处理对应的actor的消息处理上。与此同时,其他的流管理功能可以以消息处理函数的形式被添加到流actor上。%但是,已有的actor的系统都没有基于NFV的执行环境进行优化。在我们最初版本使用了libcaf作为actor执行库,但是却无法得到一个令人满意的结果。因此我们制作了自己的actor执行模型并大幅度提升了性能。

Actor model is a natural fit when buildling flow execution context. In a VNF instance, we can create one actor for one flow, and map the flow packet processing to actor message processing. In the mean time, the flow management tasks could be implemented as message handlers on the actor. However, none of the existing actor systems are optimzed for NFV envirnoment. In our initial prototype, we use C++ Actor Framework \cite{caf} to build NFActor, but the performance of that prototype turns out to be not satisfactory. This forces us to make a customized actor model for NFActor and greatly improves the performance. 

%If we treat a NF instance as an actor, then the incoming packets could be viewed as messages that are sent to this actor's mailbox. Processing packets could be mapped to handling messages using NF software's message handler. Even though there is a simple and clear relationship between NF instances and actor model, there has been no attempt to build NFV system using actor model.
