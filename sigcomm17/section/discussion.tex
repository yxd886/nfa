\section{Discussion}
\label{sec:discussion}

Even though~\nfactor provides transparent resilience for stateful NFs,~\nfactor focuses on handling per-flow state. Currently,~\nfactor could not correctly handle shared states, \ie, the states shared by a bunch of flows. Even though the NF API in~\nfactor achieves a clean separation between per-flow state and NF processing logic, it can not correctly separate shared state. Therefore, migrating and replicating flows that share states with other flows may cause un-predicted errors in~\nfactor. A potential solution to this limitation is to enforce the programmer to write a handler that explicitly deals with the inconsistency during resilience operation. We leave this to our future work.

Another limitation of~\nfactor is that~\nfactor may incorrectly handle flows with packet encapsulation. \nfactor uses the flow-5-tuple to differentiate flows. However, different flows may share the same flow-5-tuple if their flow packets are encapsulated. This is a common for flows that are sent over the same VxLAN tunnel. In that case, those flows are handled by the same flow actor, resulting in incorrect flow processing. If~\nfactor knows what kind of encapsulation the input packet uses,~\nfactor could add a decapsulation function in the virtual switch to correctly extract different flows. This is also left in our future work.

To achieve transparent resilience,~\nfactor~requires NF to be rewritten a new set of API to achieve clean separation between flow state and NF core logic, making legacy NFs difficult to run on~\nfactor. However, with the development of NFV system, there is a practical need for people to create new NFs. NFs that process flows based on flow state could achieve transparent resilient if they are implemented using~\nfactor.
